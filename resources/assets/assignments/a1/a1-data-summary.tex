\documentclass[letterpaper,11pt,twoside,]{pinp}

%% Some pieces required from the pandoc template
\providecommand{\tightlist}{%
  \setlength{\itemsep}{0pt}\setlength{\parskip}{0pt}}

% Use the lineno option to display guide line numbers if required.
% Note that the use of elements such as single-column equations
% may affect the guide line number alignment.

\usepackage[T1]{fontenc}
\usepackage[utf8]{inputenc}

% pinp change: the geometry package layout settings need to be set here, not in pinp.cls
\geometry{layoutsize={0.95588\paperwidth,0.98864\paperheight},%
  layouthoffset=0.02206\paperwidth, layoutvoffset=0.00568\paperheight}

\definecolor{pinpblue}{HTML}{185FAF}  % imagecolorpicker on blue for new R logo
\definecolor{pnasbluetext}{RGB}{101,0,0} %



\title{Assignment 2 - Data Visualization. Due September 22, 11:59pm 2019}

\author[a]{EPIB607 - Inferential Statistics}

  \affil[a]{Fall 2019, McGill University}

\setcounter{secnumdepth}{5}

% Please give the surname of the lead author for the running footer
\leadauthor{Bhatnagar and Hanley}

% Keywords are not mandatory, but authors are strongly encouraged to provide them. If provided, please include two to five keywords, separated by the pipe symbol, e.g:
 \keywords{  Histograms |  Bar plots |  Line plots |  ggformula package |  mosaic package  }  

\begin{abstract}
The first step in understanding data is to hear what the data say, to
``let the data speak for themselves''. Numbers speak clearly only when
we help them speak by organizing, displaying, and summarizing. In this
assignment you will explore how to visualize your data and critique
figures from published papers. These papers are from the
\href{https://sahirbhatnagar.com/EPIB607/terms-and-concepts.html}{`Terms
and Concepts'} in-class exercise. All questions are to be answered in an
R Markdown document using the provided template. You are free to choose
any function from any package to complete the assignment. Concise
answers will be rewarded. Be brief and to the point. Each question is
worth 25 points. There will be a bonus of 5 points if your assignment is
reproducible. Label your graphs appropriately with proper titles and
axis labels. Please submit both the compiled HTML report and the source
file (.Rmd) to myCourses by September 22, 2019, 11:59pm. Both HTML and
.Rmd files should be saved as
`IDnumber\_LastName\_FirstName\_EPIB607\_A2'.
\end{abstract}

\dates{This version was compiled on \today} 

% initially we use doi so keep for backwards compatibility
\doifooter{\url{https://sahirbhatnagar.com/EPIB607/}}
% new name is doi_footer

\pinpfootercontents{Assignment 2 due Sepetember 22, 2019 by 11:59pm}

\begin{document}

% Optional adjustment to line up main text (after abstract) of first page with line numbers, when using both lineno and twocolumn options.
% You should only change this length when you've finalised the article contents.
\verticaladjustment{-2pt}

\maketitle
\thispagestyle{firststyle}
\ifthenelse{\boolean{shortarticle}}{\ifthenelse{\boolean{singlecolumn}}{\abscontentformatted}{\abscontent}}{}

% If your first paragraph (i.e. with the \dropcap) contains a list environment (quote, quotation, theorem, definition, enumerate, itemize...), the line after the list may have some extra indentation. If this is the case, add \parshape=0 to the end of the list environment.


\hypertarget{template}{%
\section*{Template}\label{template}}
\addcontentsline{toc}{section}{Template}

The \texttt{.Rmd} template for Assignment 2 is available
\href{https://github.com/sahirbhatnagar/EPIB607/raw/master/assignments/a2/a2_template.Rmd}{here}

\hypertarget{the-mosaic-package-optional}{%
\section*{\texorpdfstring{The \texttt{mosaic} package
(optional)}{The mosaic package (optional)}}\label{the-mosaic-package-optional}}
\addcontentsline{toc}{section}{The \texttt{mosaic} package (optional)}

The \texttt{mosaic} package provides a consistent and user-friendly
interface for descriptive statistics, plots and inference. You may find
it useful to complete an interactive tutorial on its plotting functions.
(note: this is optional and will not be counted for any marks). First
install the following packages:

\begin{Shaded}
\begin{Highlighting}[]
\KeywordTok{install.packages}\NormalTok{(}\KeywordTok{c}\NormalTok{(}\StringTok{"pacman"}\NormalTok{, }\StringTok{"mosaic"}\NormalTok{), }\DataTypeTok{dependencies =} \OtherTok{TRUE}\NormalTok{)}
\NormalTok{pacman}\OperatorTok{::}\KeywordTok{p_install_gh}\NormalTok{(}\StringTok{"rstudio/learnr"}\NormalTok{)}
\end{Highlighting}
\end{Shaded}

Then, from RStudio, run the following command which will open a new page
in your web broswer:

\begin{Shaded}
\begin{Highlighting}[]
\NormalTok{learnr}\OperatorTok{::}\KeywordTok{run_tutorial}\NormalTok{(}\StringTok{"introduction"}\NormalTok{, }\DataTypeTok{package =} \StringTok{"ggformula"}\NormalTok{)}
\end{Highlighting}
\end{Shaded}

An advanced tutorial on customizing plots is available also:

\begin{Shaded}
\begin{Highlighting}[]
\NormalTok{learnr}\OperatorTok{::}\KeywordTok{run_tutorial}\NormalTok{(}\StringTok{"refining"}\NormalTok{, }\DataTypeTok{package =} \StringTok{"ggformula"}\NormalTok{)}
\end{Highlighting}
\end{Shaded}

\hypertarget{indoor-tanning-among-us-high-school-students}{%
\section{Indoor Tanning Among US High School
Students}\label{indoor-tanning-among-us-high-school-students}}

Read the paper by Qin et al.~(2018) \emph{State Indoor Tanning Laws and
Prevalence of Indoor Tanning Among US High School Students,2009--2015}
available
\href{https://sahirbhatnagar.com/EPIB607/terms-and-concepts.html}{here}

\begin{enumerate}
\def\labelenumi{\alph{enumi})}
\tightlist
\item
  (8 points) Consider Figure 1: What visual cues (or aesthetics) are
  being used? Briefly describe the main takeaways from Plot (b).
\item
  (5 points) Do you think the two graphs are clear? Is there anything
  you would have done differently?
\item
  (12 points) Consider Table 1: produce a plot that displays the indoor
  tanning prevalence and confidence intervals among female students by
  restriction type (no restriction, age restriction and parental
  permission) between 2009 and 2015 (you have to manually enter the
  data).
\end{enumerate}

\hypertarget{folate-nutrition-status-in-mothers-of-the-boston-birth-cohort-sample-of-a-us-urban-low-income-population}{%
\section{Folate Nutrition Status in Mothers of the Boston Birth Cohort,
Sample of a US Urban Low-Income
Population}\label{folate-nutrition-status-in-mothers-of-the-boston-birth-cohort-sample-of-a-us-urban-low-income-population}}

\begin{enumerate}
\def\labelenumi{\alph{enumi})}
\item
  (5 points) Table 2 shows the folate intake frequncy across different
  pregnancy stages. Here, we will visualize a part of information shown
  in this Table. For the \texttt{Total\ Sample\ (n=7612)} data only,
  create a summary-level \texttt{data.frame} that has 3 columns:
  (pregnancy) \textbf{period}, (folate intake) \textbf{frequency}
  (including `missing' field), and \textbf{count} (of total sample).
  Alternatively, you can create an individual-level data frame that has
  2 columns: (pregnancy) \textbf{period} and (folate intake)
  \textbf{frequency}, repeating each combination of the two variables by
  their corresponding sample counts.
\item
  (8 points) Display the data by a grouped barplot, where the folate
  intake frequency is grouped by pregnancy period, and the height of
  each bar is the corresponding sample counts. Don't forget to add an
  appropriate title and caption to the plot. What can you interpret from
  this plot?
\item
  (12 points) Another way to draw a grouped barplot is to group the
  pregnancy period by folate intake frequency. Draw this plot and
  compare it with last plot. What can you interpret from this plot?
  Briefly describe the main takeaways from each of these two plots.
\end{enumerate}

\hypertarget{flint-blood-lead-levels}{%
\section{Flint Blood Lead Levels}\label{flint-blood-lead-levels}}

Lead in the environment is persistent, bio-accumulative, and toxic.
Chronic exposure to lead in children is associated with many negative
health outcomes even when the Blood Lead Levels (BLLs) are measured as
low as 1.0-10.0 µg/dL. An analysis of childhood exposure to lead is
described in the article \emph{Blood Lead Levels of Children in Flint,
Michigan: 2006-2016}.

\begin{enumerate}
\def\labelenumi{\alph{enumi})}
\item
  (2 points) As presented in, Figure 3, is BLL on a continuous or
  discrete scale? Is Frequency on a continuous or discrete scale?
\item
  (3 points) Briefly comment on a strength of Figure 3. i.e.: what
  information jumps out at you when you look at this figure?
\item
  (3 points) Briefly comment on a weakness of Figure 3. i.e.: what
  information is presented in the figure, but is more difficult to
  interpret/see?
\item
  (17 points: 5 for data entry, 6 for visualization, 3 for advantage,
  and 3 for disadvantage) Figure 3 data in tabular form is show below.
  Create a \texttt{data.frame} of the data and a new visualization using
  the data. Explain why your data visualization is better and/or worse
  than Figure 3. The visualization does not need to be radically
  different than Figure 3, but it can be if you are feeling especially
  rad today. The aim is to make a visualization that is \emph{not} a bar
  chart and to explain at least one advantage and one disadvantage of
  your new and exciting visualization with respect to Figure 3. You may
  facet the visualization
  (e.g.~\url{https://cran.r-project.org/web/packages/ggformula/vignettes/ggformula-blog.html\#facets})
  as long as you justify your choice. In the data shown below, the
  \texttt{bll\_cat} column is the category of BLL and \texttt{cat\_freq}
  is the frequency percentage of the particular BLL category in a given
  year. All values are taken from Figure 3 and the values for
  \texttt{bll\_cat} are approximate.
\end{enumerate}

\begin{ShadedResult}
\begin{verbatim}
#     year bll_cat cat_freq
#  1  2012    <0.5      2.0
#  2  2013    <0.5      4.5
#  3  2014    <0.5      6.0
#  4  2015    <0.5      4.5
#  5  2016    <0.5      6.0
#  6  2012 0.5-1.0     28.5
#  7  2013 0.5-1.0     37.0
#  8  2014 0.5-1.0     35.0
#  9  2015 0.5-1.0     35.0
#  10 2016 0.5-1.0     44.5
#  11 2012 1.1-2.0     45.0
#  12 2013 1.1-2.0     38.0
#  13 2014 1.1-2.0     33.0
#  14 2015 1.1-2.0     42.0
#  15 2016 1.1-2.0     36.0
#  16 2012 2.1-3.0     14.0
#  17 2013 2.1-3.0     12.0
#  18 2014 2.1-3.0      9.0
#  19 2015 2.1-3.0     12.5
#  20 2016 2.1-3.0      7.0
#  21 2012 3.1-4.0      5.0
#  22 2013 3.1-4.0      4.5
#  23 2014 3.1-4.0      3.0
#  24 2015 3.1-4.0      3.0
#  25 2016 3.1-4.0      2.5
#  26 2012 4.1-5.0      2.5
#  27 2013 4.1-5.0      2.5
#  28 2014 4.1-5.0      2.0
#  29 2015 4.1-5.0      2.0
#  30 2016 4.1-5.0      1.0
#  31 2012    >5.0      3.0
#  32 2013    >5.0      2.0
#  33 2014    >5.0      3.0
#  34 2015    >5.0      3.5
#  35 2016    >5.0      2.5
\end{verbatim}
\end{ShadedResult}

\hypertarget{physicochemical-properties-and-phenolic-content-of-honey-from-different-floral-origins-and-from-rural-versus-urban-landscapes}{%
\section{Physicochemical properties and phenolic content of honey from
different floral origins and from rural versus urban
landscapes}\label{physicochemical-properties-and-phenolic-content-of-honey-from-different-floral-origins-and-from-rural-versus-urban-landscapes}}

\begin{enumerate}
\def\labelenumi{\alph{enumi})}
\item
  (5 points) Comment on figure 2a and explain what is wrong with it.
\item
  (10 points) Which visual cues are being used in Figure 2b? Re-plot
  Figure 2b. You can make changes to the aesthetics or reproduce it. You
  do not need to reproduce the error bars or the sample size (n). Be
  sure to include labels and a title.
\item
  (5 points) Which visual cues are being used in Figure 4? Is this an
  effective visualization? Is there a better way to represent the data?
  (you may take inspiration from
  \url{https://serialmentor.com/dataviz/directory-of-visualizations.html}
  or
  \url{http://r-statistics.co/Top50-Ggplot2-Visualizations-MasterList-R-Code.html\#3.\%20Ranking})
\item
  (5 points) List errors in Table 1 and Figure 4 and what could have
  been improved.
\end{enumerate}

%\showmatmethods


\bibliography{pinp}
\bibliographystyle{jss}



\end{document}

