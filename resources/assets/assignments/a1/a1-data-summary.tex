\documentclass[letterpaper,12pt,twoside,]{pinp}

%% Some pieces required from the pandoc template
\providecommand{\tightlist}{%
  \setlength{\itemsep}{0pt}\setlength{\parskip}{0pt}}

% Use the lineno option to display guide line numbers if required.
% Note that the use of elements such as single-column equations
% may affect the guide line number alignment.

\usepackage[T1]{fontenc}
\usepackage[utf8]{inputenc}

% pinp change: the geometry package layout settings need to be set here, not in pinp.cls
\geometry{layoutsize={0.95588\paperwidth,0.98864\paperheight},%
  layouthoffset=0.02206\paperwidth, layoutvoffset=0.00568\paperheight}

\definecolor{pinpblue}{HTML}{185FAF}  % imagecolorpicker on blue for new R logo
\definecolor{pnasbluetext}{RGB}{101,0,0} %



\title{Assignment 1 - Exploring Data. Due September 25, 11:59pm 2019}

\author[a]{EPIB607 - Inferential Statistics}

  \affil[a]{Fall 2020, McGill University}

\setcounter{secnumdepth}{5}

% Please give the surname of the lead author for the running footer
\leadauthor{Bhatnagar}

% Keywords are not mandatory, but authors are strongly encouraged to provide them. If provided, please include two to five keywords, separated by the pipe symbol, e.g:
 \keywords{  Histograms |  Bar plots |  Line plots |  ggformula package |  mosaic package  }  

\begin{abstract}
All questions are to be answered in an R Markdown document using the
provided template. You are free to choose any function from any package
to complete the assignment. Concise answers will be rewarded. Be brief
and to the point. Each question is worth 25 points. Label your graphs
appropriately with proper titles and axis labels. Please submit both the
compiled HTML report and the source file (.Rmd) to myCourses by
September 22, 2019, 11:59pm. Both HTML and .Rmd files should be saved as
`IDnumber\_LastName\_FirstName\_EPIB607\_A2'.
\end{abstract}

\dates{This version was compiled on \today} 

% initially we use doi so keep for backwards compatibility
% new name is doi_footer

\pinpfootercontents{Assignment 1 due Sepetember 25, 2020 by 11:59pm}

\begin{document}

% Optional adjustment to line up main text (after abstract) of first page with line numbers, when using both lineno and twocolumn options.
% You should only change this length when you've finalised the article contents.
\verticaladjustment{-2pt}

\maketitle
\thispagestyle{firststyle}
\ifthenelse{\boolean{shortarticle}}{\ifthenelse{\boolean{singlecolumn}}{\abscontentformatted}{\abscontent}}{}

% If your first paragraph (i.e. with the \dropcap) contains a list environment (quote, quotation, theorem, definition, enumerate, itemize...), the line after the list may have some extra indentation. If this is the case, add \parshape=0 to the end of the list environment.


\hypertarget{template}{%
\section*{Template}\label{template}}
\addcontentsline{toc}{section}{Template}

Please use the \texttt{.Rmd} template for Assignment 1 is available on
myCourses.

\hypertarget{points-immunogenicity-of-the-chadox1-ncov-19-vaccine-against-sars-cov-2}{%
\section{(25 points) Immunogenicity of the ChAdOx1 nCoV-19 vaccine
against
SARS-CoV-2}\label{points-immunogenicity-of-the-chadox1-ncov-19-vaccine-against-sars-cov-2}}

This questions refers to the Lancet paper \emph{Safety and
immunogenicity of the ChAdOx1 nCoV-19 vaccine against SARS-CoV-2:a
preliminary report of a phase 1/2, single-blind, randomised controlled
trial} by Folegatti et. al (2020) and available in myCourses.

\begin{enumerate}
\def\labelenumi{\alph{enumi})}
\tightlist
\item
  (2 points) Consider Figure 3 Panel B: What visual cues (or aesthetics)
  are being used? Briefly describe the main takeaways from the entire
  Figure 3.
\item
  (3 points) Do you think Figure 3 is a good graphic in terms of
  conveying its message clearly? Is there anything you would have done
  differently? Explain.
\item
  (2 points) Consider the data introduced in class which contains
  immunity levels (Immunoglobulin G (IgG)) from the convalescent group
  and the vaccine groups post 28 days. Note that the IgG levels in the
  dataset below are given on the log10 scale. Calculate the median IgG
  levels (ELISA units) on the log10 scale for each group.
\end{enumerate}

\begin{Shaded}
\begin{Highlighting}[]
\NormalTok{path <-}
\StringTok{"http://www.biostat.mcgill.ca/hanley/statbook/immunogenicityChAdOx1.nCoV-19vaccine.txt"}
\NormalTok{ds <-}\StringTok{ }\KeywordTok{read.table}\NormalTok{(path)}
\KeywordTok{head}\NormalTok{(ds)}
\end{Highlighting}
\end{Shaded}

\begin{ShadedResult}
\begin{verbatim}
#    RefIndexCategory IgGResponse.log10.ElisaUnits
#  1     Convalescent                         2.56
#  2     Convalescent                         2.74
#  3     Convalescent                         2.79
#  4     Convalescent                         3.32
#  5     Convalescent                         3.15
#  6     Convalescent                         2.35
\end{verbatim}
\end{ShadedResult}

\begin{Shaded}
\begin{Highlighting}[]
\KeywordTok{str}\NormalTok{(ds)}
\end{Highlighting}
\end{Shaded}

\begin{ShadedResult}
\begin{verbatim}
#  'data.frame':    307 obs. of  2 variables:
#   $ RefIndexCategory            : Factor w/ 2 levels "Convalescent",..: 1 1 1 1 1 1 1 1 1 1 ...
#   $ IgGResponse.log10.ElisaUnits: num  2.56 2.74 2.79 3.32 3.15 2.35 2.72 2.95 2.42 2.64 ...
\end{verbatim}
\end{ShadedResult}

\begin{enumerate}
\def\labelenumi{\alph{enumi})}
\setcounter{enumi}{3}
\tightlist
\item
  (4 points) From the medians alone, is there enough evidence to
  conclude that the median IgG levels in the convalescent group are
  higher than the median IgG levels in the vaccine group (post 28 days)?
  Explain.
\item
  (7 points) Use the Boostrap to asses if there is enough evidence to
  suggest that the median IgG levels in the convalescent group are
  higher than the median IgG levels in the vaccine group (post 28 days).
  \emph{Hint: resample the data with replacement separately in each
  group B=1000 times. For each of the B datasets, calculate the median
  IgG level and take the difference in medians between the two groups.
  Plot the differences in a histogram and calculate the 2.5 and 97.5
  percentiles.}
\item
  (7 points) The dataset, shown below and available on myCourses, was
  extracted (approximately) from Figure 3 Panel A for the ChAdOx1
  nCoV-19 (prime) group only. The \texttt{time} column represents the
  days since vaccination, and \texttt{igg\_response} are the IgG levels
  on the original scale. Create an appropriate figure which shows the
  immunity levels as a function of time. You are free to choose the plot
  type; the choice of plot should be guided by the message you are
  trying to convey. Be sure to label your axes, show units, include a
  title and choose an appropriate color palette. Briefly interpret the
  plot.
\end{enumerate}

\begin{Shaded}
\begin{Highlighting}[]
\NormalTok{DT <-}\StringTok{ }\KeywordTok{read.csv}\NormalTok{(}\StringTok{"prime_igg_response.csv"}\NormalTok{)}
\end{Highlighting}
\end{Shaded}

\begin{Shaded}
\begin{Highlighting}[]
\KeywordTok{head}\NormalTok{(DT)}
\end{Highlighting}
\end{Shaded}

\begin{ShadedResult}
\begin{verbatim}
#    time igg_response
#  1    0    930.37376
#  2    0    267.80142
#  3    0    241.40290
#  4    0    170.80787
#  5    0     79.79795
#  6    0     67.12348
\end{verbatim}
\end{ShadedResult}

\newpage

\hypertarget{points-covid-19-cases-comparison-between-counties-with-and-without-stay-at-home-orders}{%
\section{(25 points) COVID-19 Cases Comparison Between Counties With and
Without Stay-at-Home
Orders}\label{points-covid-19-cases-comparison-between-counties-with-and-without-stay-at-home-orders}}

This question is based on the JAMA Network Open article \emph{Comparison
of Estimated Rates of Coronavirus Disease 2019 (COVID-19) in Border
Counties in Iowa Without a Stay-at-Home Order and Border Counties in
Illinois With a Stay-at-Home Order} by Lyu and Wehby (2020) and
available in myCourses. The county and state level cumulative incidence
of cases data is provided in the code below. Note: you need to install
the \texttt{covdata} package (which is not on CRAN) using
\texttt{remotes::install\_github("kjhealy/covdata")}.

\begin{Shaded}
\begin{Highlighting}[]
\CommentTok{# remotes::install_github("kjhealy/covdata")}
\KeywordTok{library}\NormalTok{(covdata) }
\KeywordTok{library}\NormalTok{(dplyr); }\KeywordTok{library}\NormalTok{(tidyr); }\KeywordTok{library}\NormalTok{(ggplot2); }\KeywordTok{library}\NormalTok{(readr)}
\CommentTok{# get population data from https://covid19.census.gov/datasets/}
\NormalTok{f <-}\StringTok{ "https://opendata.arcgis.com/datasets/21843f238cbb46b08615fc53e19e0daf_1.csv"}
\NormalTok{pop_county <-}\StringTok{ }\KeywordTok{read_csv}\NormalTok{(}\DataTypeTok{file =}\NormalTok{ f) }\OperatorTok
\StringTok{  }\NormalTok{dplyr}\OperatorTok{::}\KeywordTok{rename}\NormalTok{(}\DataTypeTok{fips =}\NormalTok{ GEOID, }\DataTypeTok{population =}\NormalTok{ B01001_001E, }\DataTypeTok{state =}\NormalTok{ State) }\OperatorTok
\StringTok{  }\NormalTok{dplyr}\OperatorTok{::}\KeywordTok{select}\NormalTok{(state, fips, population)}

\NormalTok{county_level <-}\StringTok{ }\NormalTok{nytcovcounty }\OperatorTok
\StringTok{  }\NormalTok{dplyr}\OperatorTok{::}\KeywordTok{left_join}\NormalTok{(pop_county, }\DataTypeTok{by =} \KeywordTok{c}\NormalTok{(}\StringTok{"state"}\NormalTok{,}\StringTok{"fips"}\NormalTok{)) }\OperatorTok
\StringTok{  }\NormalTok{dplyr}\OperatorTok{::}\KeywordTok{mutate}\NormalTok{(}\DataTypeTok{cases.per.10k =}\NormalTok{ cases}\OperatorTok{/}\NormalTok{population }\OperatorTok{*}\StringTok{ }\FloatTok{1e4}\NormalTok{) }\OperatorTok
\StringTok{  }\NormalTok{dplyr}\OperatorTok{::}\KeywordTok{filter}\NormalTok{(state }\OperatorTok\StringTok{ }\KeywordTok{c}\NormalTok{(}\StringTok{"Iowa"}\NormalTok{,}\StringTok{"Illinois"}\NormalTok{)) }\OperatorTok
\StringTok{  }\NormalTok{dplyr}\OperatorTok{::}\KeywordTok{group_by}\NormalTok{(county)}

\NormalTok{pop_state <-}\StringTok{ }\NormalTok{pop_county }\OperatorTok
\StringTok{  }\NormalTok{dplyr}\OperatorTok{::}\KeywordTok{group_by}\NormalTok{(state) }\OperatorTok
\StringTok{  }\NormalTok{dplyr}\OperatorTok{::}\KeywordTok{summarise}\NormalTok{(}\DataTypeTok{population =} \KeywordTok{sum}\NormalTok{(population, }\DataTypeTok{na.rm =} \OtherTok{TRUE}\NormalTok{))}

\NormalTok{state_level <-}\StringTok{ }\NormalTok{county_level }\OperatorTok
\StringTok{  }\NormalTok{dplyr}\OperatorTok{::}\KeywordTok{group_by}\NormalTok{(state, date) }\OperatorTok
\StringTok{  }\NormalTok{dplyr}\OperatorTok{::}\KeywordTok{filter}\NormalTok{(date }\OperatorTok{>=}\StringTok{ "2020-03-15"}\NormalTok{) }\OperatorTok
\StringTok{  }\NormalTok{dplyr}\OperatorTok{::}\KeywordTok{summarise}\NormalTok{(}\DataTypeTok{cases =} \KeywordTok{sum}\NormalTok{(cases)) }\OperatorTok
\StringTok{  }\NormalTok{dplyr}\OperatorTok{::}\KeywordTok{left_join}\NormalTok{(pop_state, }\DataTypeTok{by =} \StringTok{"state"}\NormalTok{) }\OperatorTok
\StringTok{  }\NormalTok{dplyr}\OperatorTok{::}\KeywordTok{mutate}\NormalTok{(}\DataTypeTok{cases.per.10k =}\NormalTok{ cases }\OperatorTok{/}\StringTok{ }\NormalTok{population }\OperatorTok{*}\StringTok{ }\FloatTok{1e4}\NormalTok{, }\DataTypeTok{state =} \KeywordTok{factor}\NormalTok{(state),}
                \DataTypeTok{time =} \KeywordTok{as.numeric}\NormalTok{(date }\OperatorTok{-}\StringTok{ }\KeywordTok{min}\NormalTok{(date)) }\OperatorTok{+}\StringTok{ }\DecValTok{1}\NormalTok{)}
\end{Highlighting}
\end{Shaded}

\begin{Shaded}
\begin{Highlighting}[]
\KeywordTok{head}\NormalTok{(county_level)}
\end{Highlighting}
\end{Shaded}

\begin{ShadedResult}
\begin{verbatim}
#  # A tibble: 6 x 8
#  # Groups:   county [1]
#    date       county state    fips  cases deaths population cases.per.10k
#    <date>     <chr>  <chr>    <chr> <dbl>  <dbl>      <dbl>         <dbl>
#  1 2020-01-24 Cook   Illinois 17031     1      0    5223719       0.00191
#  2 2020-01-25 Cook   Illinois 17031     1      0    5223719       0.00191
#  3 2020-01-26 Cook   Illinois 17031     1      0    5223719       0.00191
#  4 2020-01-27 Cook   Illinois 17031     1      0    5223719       0.00191
#  5 2020-01-28 Cook   Illinois 17031     1      0    5223719       0.00191
#  6 2020-01-29 Cook   Illinois 17031     1      0    5223719       0.00191
\end{verbatim}
\end{ShadedResult}

\begin{Shaded}
\begin{Highlighting}[]
\KeywordTok{head}\NormalTok{(state_level)}
\end{Highlighting}
\end{Shaded}

\begin{ShadedResult}
\begin{verbatim}
#  # A tibble: 6 x 6
#  # Groups:   state [1]
#    state    date       cases population cases.per.10k  time
#    <fct>    <date>     <dbl>      <dbl>         <dbl> <dbl>
#  1 Illinois 2020-03-15    94   12821497        0.0733     1
#  2 Illinois 2020-03-16   104   12821497        0.0811     2
#  3 Illinois 2020-03-17   159   12821497        0.124      3
#  4 Illinois 2020-03-18   286   12821497        0.223      4
#  5 Illinois 2020-03-19   420   12821497        0.328      5
#  6 Illinois 2020-03-20   583   12821497        0.455      6
\end{verbatim}
\end{ShadedResult}

\begin{enumerate}
\def\labelenumi{\alph{enumi})}
\tightlist
\item
  (6 points) Using the county level dataset provided, reproduce Figure 1
  of the paper. Does your Figure agree with theirs? Would county level
  curves have been more appropriate to show instead of the state totals?
\item
  (5 points) Plot the cumulative incidence curves per 10000 people from
  March 21 until the most recent day for which you have data, for each
  of the counties used in the paper. Interpret the plot and discuss if
  the county level plots still agree with the overall conclusion of the
  paper.
\item
  (4 points) Case counts are inherently tied to testing capacity. Death
  from COVID19 doesn't have this issue, although there are other biases
  such as misclassification and under reporting. Plot the same graph as
  in part (b) but for deaths and interpret the plot.
\item
  (10 points) Illinois (Democrat-controlled legislature) is surrounded
  by states with
  \href{https://www.ncsl.org/research/about-state-legislatures/partisan-composition.aspx\#}{Republican-controlled
  legislatures (Iowa, Missouri, Kentucky, Indianna, Wisconsin)}. Do the
  data suggest there is a correlation between COVID-19 cases (or deaths)
  and which party has legislative control? Explain and justify using
  summary statistics and/or figures.
\end{enumerate}

\newpage

\hypertarget{points-age-structures-of-populations-then-and-now}{%
\section{(25 points) Age-structures of Populations, then and
now}\label{points-age-structures-of-populations-then-and-now}}

The 1911 census of Ireland was taken on April 2nd 1911 and was released
to the public in 1961. Follow
\href{http://www.census.nationalarchives.ie/help/about19011911census.html}{this
link} for further details on the census. James Hanley (JH) has scrapped
the data for Dublin, collected the age-frequency distribtion by gender
and provided you with a three column .csv file on myCourses called
\texttt{age\_sex\_frequencies\_ireland.csv} which looks like this:

\begin{Shaded}
\begin{Highlighting}[]
\NormalTok{cens <-}\StringTok{ }\KeywordTok{read.csv}\NormalTok{(}\StringTok{"age_sex_frequencies_ireland.csv"}\NormalTok{)}
\end{Highlighting}
\end{Shaded}

\begin{Shaded}
\begin{Highlighting}[]
\KeywordTok{head}\NormalTok{(cens)}
\end{Highlighting}
\end{Shaded}

\begin{ShadedResult}
\begin{verbatim}
#    Gender Age Freq
#  1   Male   0 5332
#  2   Male   1 4570
#  3   Male   2 4979
#  4   Male   3 4789
#  5   Male   4 4884
#  6   Male   5 4787
\end{verbatim}
\end{ShadedResult}

The \texttt{Age} column represents the age in 1911. The \texttt{Freq}
column gives the frequency of the number of people for a given age and
\texttt{Gender}. Note that \texttt{Age} is an interval; for example,
\texttt{Age=0} actually represents individuals who are between the ages
of 0 and 1, \texttt{Age=1} are individuals between ages 1 and 2, and so
on.

\begin{enumerate}
\def\labelenumi{\alph{enumi})}
\tightlist
\item
  (6 points) What was the earliest year of birth for (i) males and (ii)
  females ?
\item
  (8 points) Create a suitable visualization of this data and then
  comment on any patterns you see and give reasons for these patterns.
  Your choice should leverage all the information provided in the data
  and be influenced by the message that you are trying to convey. Be
  sure to include an informative title and figure caption.
\item
  (8 points) Calculate the mean age, the standard deviation (SD), and
  the quartiles: \(Q_{25}, Q_{50} (median), Q_{75}\) separately for
  males and females.
\item
  (3 points) The original census cards have been scanned are available
  online.
  \href{http://www.census.nationalarchives.ie/reels/nai000230598/}{This
  one in particular} is quite famous. Why?
\end{enumerate}

\newpage

\hypertarget{points-flint-blood-lead-levels}{%
\section{(25 points) Flint Blood Lead
Levels}\label{points-flint-blood-lead-levels}}

Lead in the environment is persistent, bio-accumulative, and toxic.
Chronic exposure to lead in children is associated with many negative
health outcomes even when the Blood Lead Levels (BLLs) are measured as
low as 1.0-10.0 µg/dL. An analysis of childhood exposure to lead is
described in the article \emph{Blood Lead Levels of Children in Flint,
Michigan: 2006-2016} by Gomez et al.~(2018) available on myCourses.

\begin{enumerate}
\def\labelenumi{\alph{enumi})}
\item
  (2 points) As presented in, Figure 3, is BLL on a continuous or
  discrete scale? Is Frequency on a continuous or discrete scale?
\item
  (3 points) Briefly comment on a strength of Figure 3. i.e.: what
  information jumps out at you when you look at this figure?
\item
  (3 points) Briefly comment on a weakness of Figure 3. i.e.: what
  information is presented in the figure, but is more difficult to
  interpret/see?
\item
  (17 points: 5 for data entry, 6 for visualization, 3 for advantage,
  and 3 for disadvantage) Figure 3 data in tabular form is show below.
  Create a \texttt{data.frame} of the data and a new visualization using
  the data. Explain why your data visualization is better and/or worse
  than Figure 3. The visualization does not need to be radically
  different than Figure 3, but it can be if you are feeling especially
  rad today. The aim is to make a visualization that is \emph{not} a bar
  chart and to explain at least one advantage and one disadvantage of
  your new and exciting visualization with respect to Figure 3. You may
  \href{https://ggplot2.tidyverse.org/reference/facet_grid.html}{facet
  the visualization} as long as you justify your choice. In the data
  shown below, the \texttt{bll\_cat} column is the category of BLL and
  \texttt{cat\_freq} is the frequency percentage of the particular BLL
  category in a given year. All values are taken from Figure 3 and the
  values for \texttt{cat\_freq} are approximate. \emph{Note that we have
  not provided you with this dataset. Part of this exercise is for you
  to practice how to extract data from a graph and enter it in
  \texttt{R}.}
\end{enumerate}

\begin{Shaded}
\begin{Highlighting}[]
\KeywordTok{head}\NormalTok{(bll_df, }\DataTypeTok{n=} \DecValTok{4}\NormalTok{)}
\end{Highlighting}
\end{Shaded}

\begin{ShadedResult}
\begin{verbatim}
#    year bll_cat cat_freq
#  1 2012    <0.5      2.0
#  2 2013    <0.5      4.5
#  3 2014    <0.5      6.0
#  4 2015    <0.5      4.5
\end{verbatim}
\end{ShadedResult}

\begin{Shaded}
\begin{Highlighting}[]
\KeywordTok{str}\NormalTok{(bll_df)}
\end{Highlighting}
\end{Shaded}

\begin{ShadedResult}
\begin{verbatim}
#  'data.frame':    35 obs. of  3 variables:
#   $ year    : int  2012 2013 2014 2015 2016 2012 2013 2014 2015 2016 ...
#   $ bll_cat : Factor w/ 7 levels "<0.5",">5.0",..: 1 1 1 1 1 3 3 3 3 3 ...
#   $ cat_freq: num  2 4.5 6 4.5 6 28.5 37 35 35 44.5 ...
\end{verbatim}
\end{ShadedResult}

\begin{Shaded}
\begin{Highlighting}[]
\KeywordTok{levels}\NormalTok{(bll_df}\OperatorTok{$}\NormalTok{bll_cat)}
\end{Highlighting}
\end{Shaded}

\begin{ShadedResult}
\begin{verbatim}
#  [1] "<0.5"    ">5.0"    "0.5-1.0" "1.1-2.0" "2.1-3.0" "3.1-4.0" "4.1-5.0"
\end{verbatim}
\end{ShadedResult}

%\showmatmethods


\bibliography{pinp}
\bibliographystyle{jss}



\end{document}

