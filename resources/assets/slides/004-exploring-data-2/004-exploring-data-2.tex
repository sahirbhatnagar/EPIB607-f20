\documentclass[10pt]{beamer}\usepackage[]{graphicx}\usepackage[]{color}
% maxwidth is the original width if it is less than linewidth
% otherwise use linewidth (to make sure the graphics do not exceed the margin)
\makeatletter
\def\maxwidth{ %
  \ifdim\Gin@nat@width>\linewidth
    \linewidth
  \else
    \Gin@nat@width
  \fi
}
\makeatother

\definecolor{fgcolor}{rgb}{0.345, 0.345, 0.345}
\newcommand{\hlnum}[1]{\textcolor[rgb]{0.686,0.059,0.569}{#1}}%
\newcommand{\hlstr}[1]{\textcolor[rgb]{0.192,0.494,0.8}{#1}}%
\newcommand{\hlcom}[1]{\textcolor[rgb]{0.678,0.584,0.686}{\textit{#1}}}%
\newcommand{\hlopt}[1]{\textcolor[rgb]{0,0,0}{#1}}%
\newcommand{\hlstd}[1]{\textcolor[rgb]{0.345,0.345,0.345}{#1}}%
\newcommand{\hlkwa}[1]{\textcolor[rgb]{0.161,0.373,0.58}{\textbf{#1}}}%
\newcommand{\hlkwb}[1]{\textcolor[rgb]{0.69,0.353,0.396}{#1}}%
\newcommand{\hlkwc}[1]{\textcolor[rgb]{0.333,0.667,0.333}{#1}}%
\newcommand{\hlkwd}[1]{\textcolor[rgb]{0.737,0.353,0.396}{\textbf{#1}}}%
\let\hlipl\hlkwb

\usepackage{framed}
\makeatletter
\newenvironment{kframe}{%
 \def\at@end@of@kframe{}%
 \ifinner\ifhmode%
  \def\at@end@of@kframe{\end{minipage}}%
  \begin{minipage}{\columnwidth}%
 \fi\fi%
 \def\FrameCommand##1{\hskip\@totalleftmargin \hskip-\fboxsep
 \colorbox{shadecolor}{##1}\hskip-\fboxsep
     % There is no \\@totalrightmargin, so:
     \hskip-\linewidth \hskip-\@totalleftmargin \hskip\columnwidth}%
 \MakeFramed {\advance\hsize-\width
   \@totalleftmargin\z@ \linewidth\hsize
   \@setminipage}}%
 {\par\unskip\endMakeFramed%
 \at@end@of@kframe}
\makeatother

\definecolor{shadecolor}{rgb}{.97, .97, .97}
\definecolor{messagecolor}{rgb}{0, 0, 0}
\definecolor{warningcolor}{rgb}{1, 0, 1}
\definecolor{errorcolor}{rgb}{1, 0, 0}
\newenvironment{knitrout}{}{} % an empty environment to be redefined in TeX

\usepackage{alltt}


%\usepackage{default}
\usepackage{animate} %need the animate.sty file 
\usepackage{graphicx}
%\graphicspath{{/home/sahir/Dropbox/jobs/laval/minicours/slides/}}
\usepackage{hyperref, url}
%\usepackage[round,sort]{natbib}   % bibliography omit 'round' option if you prefer square brackets
%\bibliographystyle{apalike}
\usepackage{biblatex}
\bibliography{bib.bib}
% Removes icon in bibliography
\setbeamertemplate{bibliography item}[text]

\usepackage[normalem]{ulem}

\setbeamertemplate{theorems}[numbered]

\setbeamertemplate{caption}[numbered]
\setbeamertemplate{caption label separator}{: }
\setbeamercolor{caption name}{fg=normal text.fg}

%\newtheorem{prop}{Proposition}
%\newenvironment{theoremc}[1]
%{\begin{shaded}\begin{theorem}[#1]}
%		{\end{theorem}\end{shaded}}
	
%\newtheorem{examplefirst}{Example}
%\newtheorem{examplesecond}{Example}
%\newenvironment<>{examplefirst}[1][]{%
%	\setbeamercolor{block title example}{bg=lightgray}%
%	\begin{example}#2[#1]}{\end{example}}
%\newenvironment<>{examplesecond}[1][]{%
%	\setbeamercolor{block title example}{fg=white,bg=blue!75!black}%
%	\begin{example}#2[#1]}{\end{example}}	

%\usepackage{amsthm}


\usepackage[figurename=Fig.]{caption}
\usepackage{subfig}
\usepackage{tikz, pgfplots,epsfig}
\usetikzlibrary{arrows,shapes.geometric}
\usepackage{color, colortbl,xcolor}
\definecolor{lightgray}{RGB}{200,200,200}
\definecolor{palegray}{RGB}{221,221,221}
\definecolor{myblue}{RGB}{0,89,179}
\definecolor{myorange}{rgb}{0.776,0.357,0.157}

\definecolor{gray}{RGB}{110,110,110}
\definecolor{darkgray}{RGB}{100,100,100}
\definecolor{lightgray}{RGB}{200,200,200}
\definecolor{turquoise}{RGB}{81,193,188}
\definecolor{tomato}{RGB}{255,136,136}
\definecolor{mandarina}{RGB}{229,169,25}
\definecolor{foreground}{RGB}{81,141,193}
\definecolor{background}{RGB}{246,244,240}
\definecolor{highlight}{RGB}{229,169,25}
\definecolor{lowlight}{RGB}{200,200,200}
\definecolor{beige}{RGB}{255,255,240}
\definecolor{pinkish}{RGB}{255,223,247}

\newcommand{\code}[1]{\texttt{#1}}


\usepackage{comment}
\setbeamercolor{frametitle}{fg=myblue}
\setbeamercolor{section in head/foot}{bg=myblue, fg=white}
\setbeamercolor{author in head/foot}{bg=myblue}
\setbeamercolor{date in head/foot}{bg=myblue}

\usepackage{amsthm}
\usepackage{shadethm}
%\colorlet{shadecolor}{blue!15}
\colorlet{shadecolor}{palegray}
%\setlength{\shadeboxrule}{.4pt}

%\newshadetheorem{thm}{Theorem}
\newshadetheorem{defm}{Definition}
\newshadetheorem{exm}{Exercise}
\newshadetheorem{remarkm}{Remark}
%\definecolor{shadethmcolor}{HTML}{EDF8FF}
\definecolor{shadethmcolor}{RGB}{221,221,221}
%\definecolor{shaderulecolor}{HTML}{45CFFF}
\definecolor{shaderulecolor}{RGB}{0,89,179}
\setlength{\shadeboxrule}{.4pt}

\newtheorem{thm}{Theorem}
\newcommand{\statetheoremhoriz}[2][\textwidth]{
	\par\noindent\tikzstyle{mybox} = [draw=myblue,left color=cyan!50,
	right color=cyan!5,thick,rectangle,inner sep=6pt]
	\begin{tikzpicture}
	\node [mybox] (box){%
		\begin{minipage}{#1}{#2}\end{minipage}
	};
	\end{tikzpicture}
}
\newcommand{\statetheoremvert}[2][\textwidth]{
	\par\noindent\tikzstyle{mybox} = [draw=myblue,top color=cyan!50,
	bottom color=cyan!5,thick,rectangle,inner sep=6pt]
	\begin{tikzpicture}
	\node [mybox] (box){%
		\begin{minipage}{#1}{#2}\end{minipage}
	};
	\end{tikzpicture}
}
\newcommand{\statetheoremsolid}[2][\textwidth]{
	\par\noindent\tikzstyle{mybox} = [draw=myblue,fill=palegray,
	thick,rectangle,inner sep=6pt]
	\begin{tikzpicture}
	\node [mybox] (box){%
		\begin{minipage}{#1}{#2}\end{minipage}
	};
	\end{tikzpicture}
}

\usetikzlibrary{shapes.geometric, arrows,shapes.symbols,decorations.pathreplacing}
\tikzstyle{startstop} = [rectangle, rounded corners, minimum width=3cm, minimum height=1cm, draw=black, fill=pinkish,text width=3.5cm]
\tikzstyle{startstop2} = [rectangle, rounded corners, minimum width=3cm, minimum height=1cm, draw=black, fill=background,text width=4.5cm]
\tikzstyle{startstop3} = [rectangle, rounded corners, minimum width=3cm, minimum height=1cm, draw=black, fill=beige,text width=3.0cm]
\tikzstyle{startstop4} = [rectangle, rounded corners, minimum width=3cm, minimum height=1cm, draw=black, fill=pinkish,text width=4.5cm]
\tikzstyle{io} = [trapezium, trapezium left angle=70, trapezium right angle=110, minimum width=2cm, minimum height=1cm, text centered, draw=black, fill=blue!30,text width=1.5cm]
\tikzstyle{process} = [rectangle, minimum width=1cm, minimum height=1cm, text centered, draw=black, fill=orange!30,text width=2cm]
\tikzstyle{decision} = [diamond, minimum width=2cm, minimum height=1cm, text centered, draw=black, fill=green!30]
\tikzstyle{arrow} = [thick,->,>=stealth]
\tikzstyle{both} = [thick,<->,>=stealth, red]


\usepackage{array}
\newcolumntype{L}{>{\centering\arraybackslash}m{3cm}} % used for text wrapping in ctable
\usepackage{ctable}
\usepackage[utf8]{inputenc}
\usepackage{fontenc}
\usepackage{pifont}% http://ctan.org/pkg/pifont
\newcommand{\cmark}{\ding{51}}%
\newcommand{\xmark}{\ding{55}}%
\def\widebar#1{\overline{#1}}
\definecolor{whitesmoke}{rgb}{0.96, 0.96, 0.96}

\usepackage{amssymb}
\usepackage{amsmath}

\usepackage{bm}
\def\transpose{{\sf{T}}}
\def\E{{\skew0\bm{E}}}
\def\Xvec{{\skew0\bm{X}}}
\def\Xveca{{\skew0\bm{X}}_1}
\def\Xvecb{{\skew0\bm{X}}_2}

\def\Yvec{{\skew0\bm{Y}}}
\def\bmY{{\skew0\bm{Y}}}
\def\bmX{{\skew0\bm{X}}}
\def\bmy{{\skew0\bm{y}}}
\def\bmG{{\skew0\bm{G}}}
\def\bmS{{\skew0\bm{S}}}
\def\bmA{{\skew0\bm{A}}}
\def\bmB{{\skew0\bm{B}}}
\def\bmD{{\skew0\bm{D}}}
\def\bmI{{\skew0\bm{I}}}
\def\bmV{{\skew0\bm{V}}}
\def\bmU{{\skew0\bm{U}}}
\def\bv{{\skew0\bm{v}}}
\def\bw{{\skew0\bm{w}}}
\def\bmm{{\skew0\bm{m}}}
\def\bmzero{{\skew0\bm{0}}}
\def\bx{{\skew0\bm{x}}}
\def\xveca{{\skew0\bm{x}}_1}
\def\xvecb{{\skew0\bm{x}}_2}

\def\N{{\skew0\mathcal{N}}}
\def\T{{\small T}}

\def\mvec{{\skew0\bm{m}}}
\def\bmmu{{\skew0\bm{\mu}}}
\def\muvec{{\skew0\bm{\mu}}}
\def\balpha{{\skew0\bm{\alpha}}}
\def\bbeta{{\skew0\bm{\beta}}}
\def\bmtheta{{\skew0\bm{\theta}}}
\def\btheta{{\skew0\bm{\theta}}}

\def\cvec{{\skew0\mathbf{c}}}

\def\Xbar{\overline{X}}

\definecolor{lightgray}{rgb}{0.91,0.91,0.91}
\definecolor{purpleblue}{rgb}{0.50,0.50,1.00}



\usepackage{fontspec}
%\setsansfont{Fira Sans}
%\setmonofont{Fira Mono}
%\setsansfont[ItalicFont={Fira Sans Light Italic},BoldFont={Fira Sans},BoldItalicFont={Fira Sans Italic}]{Fira Sans Light}
%\setmonofont[BoldFont={Fira Mono Medium}]{Fira Mono}

\def\installpath{/usr/local/share/texmf/fonts/opentype/libertinus/}
\setmainfont{LibertinusSerif}[
UprightFont    = *-Regular,
BoldFont       = *-Bold,
ItalicFont     = *-Italic,
BoldItalicFont = *-BoldItalic,
Ligatures      = TeX,
Extension      = .otf,
Path           = \installpath/
]

\setsansfont{LibertinusSerif}[
UprightFont    = *-Regular,
BoldFont       = *-Bold,
ItalicFont     = *-Italic,
BoldItalicFont = *-BoldItalic,
Ligatures      = TeX,
Extension      = .otf,
Path           = \installpath/
]


\setmonofont{LibertinusSerif}[
UprightFont    = *-Regular,
BoldFont       = *-Bold,
ItalicFont     = *-Italic,
BoldItalicFont = *-BoldItalic,
Ligatures      = TeX,
Extension      = .otf,
Path           = \installpath/
]



\setbeamercolor{itemize item}{fg=myblue}
%\setbeamertemplate{itemize item}[square]
\setbeamertemplate{itemize items}[circle]
%\setbeamertemplate{blocks}[rounded][shadow=true]


%\setbeamertemplate{navigation symbols}{\usebeamercolor[fg]{title in head/foot}\usebeamerfont{title in head/foot}\insertframenumber}


%\setbeamertemplate{footline}{}

\beamertemplatenavigationsymbolsempty % toggle off if you want navigation symbols at the bottom

\setbeamertemplate{footline}
{ \usebeamercolor[fg]{page number in head/foot}%
	\usebeamerfont{page number in head/foot}%
	\hspace{1em}\insertsectionhead%
	\hfill%
	\insertframenumber\,/\,\hyperlinkappendixstart{\insertmainframenumber}
	\ifnum \thepage = \insertframeendpage{\small .}\else{\phantom{\small .}}\fi
	\hspace{1em}
	\vskip2pt%
}

\newtheorem{proposition}[theorem]{Proposition}
\newtheorem{exercise}[theorem]{Exercise}

\titlegraphic{\hfill\includegraphics[height=1cm]{../mcgill_logo.png}}


%% You also use hyperref, and pick colors 
\hypersetup{colorlinks,citecolor=myorange,filecolor=red,linkcolor=brown,urlcolor=blue}

\newcommand {\framedgraphiccaption}[2] {
	\begin{figure}
		\centering
		\includegraphics[width=\textwidth,height=0.8\textheight,keepaspectratio]{#1}
		\caption{#2}
	\end{figure}
}

\newcommand {\framedgraphic}[1] {
	\begin{figure}
		\centering
		\includegraphics[width=\textwidth,height=0.9\textheight,keepaspectratio]{#1}
	\end{figure}
}


\AtBeginSection[]{
	\begin{frame}
		\vfill
		\centering
		\begin{beamercolorbox}[sep=8pt,center,shadow=true,rounded=true]{title}
			\usebeamerfont{title}\insertsectionhead\par%
		\end{beamercolorbox}
		\vfill
	\end{frame}
}

\newcommand\Wider[2][3em]{%
	\makebox[\linewidth][c]{%
		\begin{minipage}{\dimexpr\textwidth+#1\relax}
			\raggedright#2
		\end{minipage}%
	}%
}


\makeatletter

\def \iqsssectiontitleheader {}

\newcommand{\iqsssectiontitle}[1]{
	\def \iqsssectiontitleheader{#1}
}

\@ifundefined{insertmainframenumber}
{%
	% \insertmainframenumber not defined
	\newcommand{\insertmainframenumber}{\inserttotalframenumber}
}
{%
	% \insertmainframenumber already defined
}%


\AtBeginSection[]{
	\title{\insertsectionhead}
	{
		%\definecolor{white}{RGB}{140,193,250}
		%\definecolor{white}{RGB}{200,200,200}
		%\definecolor{white}{RGB}{242,244,247}
		\definecolor{white}{RGB}{0,89,179}
		%\definecolor{iqss@orange}{rgb}{1,1,1}
		\ifnum \insertmainframenumber > \insertframenumber
		%\setbeamercolor{background canvas}{bg=myblue}
		%\setbeamercolor{normal text}{fg=black,bg=white}
		%\setbeamercolor{frametitle}{fg=red}
		%\setbeamercolor{section in toc}{fg=myblue, bg=white}
		%\setbeamercolor{subsection in toc}{fg=myblue, bg=white}
		\frame{
			\frametitle{\iqsssectiontitleheader}
			\tableofcontents[currentsection]
		}
		\else
		\frame{
			\frametitle{Backup Slides}
			\tableofcontents[sectionstyle=shaded/shaded,subsectionstyle=shaded/shaded/shaded]
		}
		\fi
	}
}
\makeatother

\pgfplotsset{compat=1.16}
\usepackage{graphicx}
\usepackage{hyperref, url}
\hypersetup{colorlinks,citecolor=myorange,filecolor=red,linkcolor=brown,urlcolor=blue}

\usepackage{subfig}
\usepackage{tikz}
\usetikzlibrary{shapes.geometric, arrows,shapes.symbols,decorations.pathreplacing}
\tikzstyle{startstop} = [rectangle, rounded corners, minimum width=3cm, minimum height=1cm, draw=black, fill=pinkish,text width=3.5cm]
\tikzstyle{startstop2} = [rectangle, rounded corners, minimum width=3cm, minimum height=1cm, draw=black, fill=background,text width=4.5cm]
\tikzstyle{startstop3} = [rectangle, rounded corners, minimum width=3cm, minimum height=1cm, draw=black, fill=beige,text width=3.0cm]
\tikzstyle{startstop4} = [rectangle, rounded corners, minimum width=3cm, minimum height=1cm, draw=black, fill=pinkish,text width=4.5cm]
\tikzstyle{io} = [trapezium, trapezium left angle=70, trapezium right angle=110, minimum width=2cm, minimum height=1cm, text centered, draw=black, fill=blue!30,text width=1.5cm]
\tikzstyle{process} = [rectangle, minimum width=1cm, minimum height=1cm, text centered, draw=black, fill=orange!30,text width=2cm]
\tikzstyle{decision} = [diamond, minimum width=2cm, minimum height=1cm, text centered, draw=black, fill=green!30]
\tikzstyle{arrow} = [thick,->,>=stealth]
\tikzstyle{both} = [thick,<->,>=stealth, red]







\usepackage{color, colortbl,xcolor}
\definecolor{lightgray}{RGB}{200,200,200}
\definecolor{palegray}{RGB}{221,221,221}
\definecolor{myblue}{RGB}{0,89,179}
\definecolor{myorange}{rgb}{0.776,0.357,0.157}
\definecolor{gray}{RGB}{110,110,110}
\definecolor{darkgray}{RGB}{100,100,100}
\definecolor{lightgray}{RGB}{200,200,200}
\definecolor{turquoise}{RGB}{81,193,188}
\definecolor{tomato}{RGB}{255,136,136}
\definecolor{mandarina}{RGB}{229,169,25}
\definecolor{foreground}{RGB}{81,141,193}
\definecolor{background}{RGB}{246,244,240}
\definecolor{highlight}{RGB}{229,169,25}
\definecolor{lowlight}{RGB}{200,200,200}
\definecolor{beige}{RGB}{255,255,240}
\definecolor{pinkish}{RGB}{255,223,247}

\newcommand{\code}[1]{\texttt{#1}}


\usepackage{comment}

\makeatletter

\def \iqsssectiontitleheader {}

\newcommand{\iqsssectiontitle}[1]{
	\def \iqsssectiontitleheader{#1}
}

\@ifundefined{insertmainframenumber}
{%
	% \insertmainframenumber not defined
	\newcommand{\insertmainframenumber}{\inserttotalframenumber}
}
{%
	% \insertmainframenumber already defined
}%


\AtBeginSection[]{
	\title{\insertsectionhead}
	{
		%\definecolor{white}{RGB}{140,193,250}
		%\definecolor{white}{RGB}{200,200,200}
		%\definecolor{white}{RGB}{242,244,247}
		\definecolor{white}{RGB}{0,89,179}
		%\definecolor{iqss@orange}{rgb}{1,1,1}
		\ifnum \insertmainframenumber > \insertframenumber
		%\setbeamercolor{background canvas}{bg=myblue}
		%\setbeamercolor{normal text}{fg=black,bg=white}
		%\setbeamercolor{frametitle}{fg=red}
		%\setbeamercolor{section in toc}{fg=myblue, bg=white}
		%\setbeamercolor{subsection in toc}{fg=myblue, bg=white}
		\frame{
			\frametitle{\iqsssectiontitleheader}
			\tableofcontents[currentsection]
		}
		\else
		\frame{
			\frametitle{Backup Slides}
			\tableofcontents[sectionstyle=shaded/shaded,subsectionstyle=shaded/shaded/shaded]
		}
		\fi
	}
}
\makeatother
%\graphicspath{{/home/sahir/git_repositories/EPIB607/resources/assets/slides/figure/}}


\usepackage{fontspec}
%\setsansfont{Fira Sans}
%\setmonofont{Fira Mono}
%\setsansfont[ItalicFont={Fira Sans Light Italic},BoldFont={Fira Sans},BoldItalicFont={Fira Sans Italic}]{Fira Sans Light}
%\setmonofont[BoldFont={Fira Mono Medium}]{Fira Mono}

\def\installpath{/usr/local/share/texmf/fonts/opentype/libertinus/}
\setmainfont{LibertinusSerif}[
UprightFont    = *-Regular,
BoldFont       = *-Bold,
ItalicFont     = *-Italic,
BoldItalicFont = *-BoldItalic,
Ligatures      = TeX,
Extension      = .otf,
Path           = \installpath/
]

\setsansfont{LibertinusSerif}[
UprightFont    = *-Regular,
BoldFont       = *-Bold,
ItalicFont     = *-Italic,
BoldItalicFont = *-BoldItalic,
Ligatures      = TeX,
Extension      = .otf,
Path           = \installpath/
]


%\setmonofont{LibertinusSerif}[
%UprightFont    = *-Regular,
%BoldFont       = *-Bold,
%ItalicFont     = *-Italic,
%BoldItalicFont = *-BoldItalic,
%Ligatures      = TeX,
%Extension      = .otf,
%Path           = \installpath/
%]






\newcommand\Wider[2][3em]{%
	\makebox[\linewidth][c]{%
		\begin{minipage}{\dimexpr\textwidth+#1\relax}
			\raggedright#2
		\end{minipage}%
	}%
}


\newcommand {\framedgraphic}[1] {
	\begin{figure}
		\centering
		\includegraphics[width=\textwidth,height=0.9\textheight,keepaspectratio]{#1}
	\end{figure}
}


\setbeamercolor{itemize item}{fg=myblue}
%\setbeamertemplate{itemize item}[square]
\setbeamertemplate{itemize items}[circle]
%\setbeamertemplate{blocks}[rounded][shadow=true]


%\setbeamertemplate{navigation symbols}{\usebeamercolor[fg]{title in head/foot}\usebeamerfont{title in head/foot}\insertframenumber}


%\setbeamertemplate{footline}{}

\beamertemplatenavigationsymbolsempty % toggle off if you want navigation symbols at the bottom

\setbeamertemplate{footline}
{ \usebeamercolor[fg]{page number in head/foot}%
	\usebeamerfont{page number in head/foot}%
	\hspace{1em}\insertsectionhead%
	\hfill%
	\insertframenumber\,/\,\hyperlinkappendixstart{\insertmainframenumber}
	\ifnum \thepage = \insertframeendpage{\small .}\else{\phantom{\small .}}\fi
	\hspace{1em}
	\vskip2pt%
}

\newtheorem{proposition}[theorem]{Proposition}
\newtheorem{exercise}[theorem]{Exercise}






\titlegraphic{\hfill\includegraphics[height=1cm]{/home/sahir/git_repositories/EPIB607/resources/assets/slides/mcgill_logo.png}}
\graphicspath{{/home/sahir/git_repositories/EPIB607/resources/assets/slides/figure/}}

%\let\oldShaded\Shaded
%\let\endoldShaded\endShaded
%\renewenvironment{Shaded}{\footnotesize\oldShaded}{\endoldShaded}
\IfFileExists{upquote.sty}{\usepackage{upquote}}{}
\begin{document}
	
	
	
	%\title{Introduction to Regression Trees}
	%\author{Sahir Bhatnagar \inst{1}}
	%\author[shortname]{Sahir Rai Bhatnagar, PhD Candidate (Biostatistics) }
	%\institute[shortinst]{Department of Epidemiology, Biostatistics and Occupational Health}
	
	\title{004 - Exploring Data - Part II}
	\author{EPIB 607 - FALL 2020}
	\institute{
		Sahir Rai Bhatnagar\\
		Department of Epidemiology, Biostatistics, and Occupational Health\\
		McGill University\\
		
		\vspace{0.1 in}
		
		\texttt{sahir.bhatnagar@mcgill.ca}\\
		%\texttt{\url{https://sahirbhatnagar.com/EPIB607/}}
	}
	
	\date{slides compiled on \today}
	
	\maketitle

	
	

						
\begin{frame}{Summarizing relationships between two variables}
							\protect\hypertarget{summarizing-relationships-between-two-variables}{}
							
							Approaches for summarizing relationships between two variables vary
							depending on variable types\ldots{}
							
							\begin{itemize}
								\item
								Two numerical variables
								\item
								Two categorical variables
								\item
								One numerical variable and one categorical variable
							\end{itemize}
							
\end{frame}


\section{Two numerical variables}


\begin{frame}[fragile]{Scatterplots}
	\protect\hypertarget{two-numerical-variables-1}{}
	
	\small
	
	\scriptsize
	
	
\begin{knitrout}\scriptsize
\definecolor{shadecolor}{rgb}{0.969, 0.969, 0.969}\color{fgcolor}\begin{kframe}
\begin{alltt}
\hlkwd{library}\hlstd{(ggplot2);} \hlkwd{library}\hlstd{(oibiostat);}
\hlkwd{data}\hlstd{(famuss)}

\hlkwd{plot}\hlstd{(famuss}\hlopt{$}\hlstd{height, famuss}\hlopt{$}\hlstd{weight,} \hlkwc{xlab} \hlstd{=} \hlstr{"Height (in)"}\hlstd{,} \hlkwc{ylab} \hlstd{=} \hlstr{"Weight (lb)"}\hlstd{)}

\hlkwd{ggplot}\hlstd{(}\hlkwc{data} \hlstd{= famuss,} \hlkwc{mapping} \hlstd{=} \hlkwd{aes}\hlstd{(}\hlkwc{x} \hlstd{= height,} \hlkwc{y} \hlstd{= weight))} \hlopt{+}
  \hlkwd{geom_point}\hlstd{(}\hlkwc{size} \hlstd{=} \hlnum{0.8}\hlstd{,} \hlkwc{pch} \hlstd{=} \hlnum{21}\hlstd{)}
\end{alltt}
\end{kframe}

{\centering \includegraphics[width=0.45\linewidth]{figure/numerical-1} 
\includegraphics[width=0.45\linewidth]{figure/numerical-2} 

}



\end{knitrout}
	
	\normalsize
	
\end{frame}

						
						\begin{frame}{Correlation coefficient}
							\protect\hypertarget{two-numerical-variables}{}
							
								\begin{itemize}
							
							\item The correlation between two variables $x$ and $y$ is given by:
							$$
							r=\frac{1}{n-1} \sum_{i=1}^{n}\left(\frac{x_{i}-\bar{x}}{s_{x}}\right)\left(\frac{y_{i}-\bar{y}}{s_{y}}\right)
							$$
							where $\left(x_{1}, y_{1}\right),\left(x_{2}, y_{2}\right), \ldots,\left(x_{n}, y_{n}\right)$ are the $n$ paired values of $x$ and $y,$ and $s_{x}$ and $s_{y}$ are the sample standard deviations of the $x$ and $y$ variables, respectively.
							
							
							\pause 
							
							\item The correlation coefficient quantifies the strength of a \textbf{\textcolor{red}{linear}} trend.
							
							\pause
							
								\item
							The correlation coefficient \(r\) takes on values between -1 and 1.
							\pause 
							\item
							The closer \(r\) is to \(\pm 1\), the stronger the linear association.
							\pause 
							
\item 	Two variables \(x\) and \(y\) are
							
							\begin{itemize}
								\item
								\emph{positively associated} if \(y\) increases as \(x\) increases ($r>0$)
								\item
								\emph{negatively associated} if \(y\) decreases as \(x\) increases ($r<0$)
							\end{itemize}
						
													\end{itemize}
							
							
						\end{frame}
						


						
\begin{frame}[fragile]{Correlation in \texttt{R}}
							\protect\hypertarget{two-numerical-variables-2}{}

							\begin{itemize}							
		\item Correlation between weight and height in the \texttt{famuss} dataset:
							
						
						
\begin{knitrout}\scriptsize
\definecolor{shadecolor}{rgb}{0.969, 0.969, 0.969}\color{fgcolor}\begin{kframe}
\begin{alltt}
\hlkwd{cor}\hlstd{(famuss}\hlopt{$}\hlstd{height, famuss}\hlopt{$}\hlstd{weight)}
\end{alltt}
\begin{verbatim}
## [1] 0.5308787
\end{verbatim}
\end{kframe}
\end{knitrout}
							
							\pause
							
							\item We can also obtain the correlation between \texttt{weight} and \texttt{height} from a simple linear regression:
							
\begin{knitrout}\scriptsize
\definecolor{shadecolor}{rgb}{0.969, 0.969, 0.969}\color{fgcolor}\begin{kframe}
\begin{alltt}
\hlkwd{summary}\hlstd{(}\hlkwd{lm}\hlstd{(height} \hlopt{~} \hlstd{weight,} \hlkwc{data} \hlstd{= famuss))}
\end{alltt}
\begin{verbatim}
## Coefficients:
##              Estimate Std. Error t value Pr(>|t|)    
## (Intercept) 58.295213   0.573200  101.70   <2e-16 ***
## weight       0.054843   0.003595   15.26   <2e-16 ***
## ---
## Signif. codes:  0 '***' 0.001 '**' 0.01 '*' 0.05 '.' 0.1 ' ' 1
## 
## Residual standard error: 3.031 on 593 degrees of freedom
## Multiple R-squared: 0.2818,	Adjusted R-squared: 0.2806 
## F-statistic: 232.7 on 1 and 593 DF,  p-value: < 2.2e-16
\end{verbatim}
\end{kframe}
\end{knitrout}
						
							\normalsize
							
					\end{itemize}
							
						\end{frame}




\begin{frame}[fragile]{title}

\vspace{-0.5in}	
	
\begin{knitrout}\tiny
\definecolor{shadecolor}{rgb}{0.969, 0.969, 0.969}\color{fgcolor}\begin{figure}

{\centering \subfloat[\label{fig:unnamed-chunk-1-1}]{\includegraphics[width=0.55\linewidth]{figure/unnamed-chunk-1-1} }
\subfloat[\label{fig:unnamed-chunk-1-2}]{\includegraphics[width=0.55\linewidth]{figure/unnamed-chunk-1-2} }

}

\caption[(a) A scatterplot showing height versus weight from the 500 individuals in the sample from NHANES]{(a) A scatterplot showing height versus weight from the 500 individuals in the sample from NHANES. One participant 163.9 cm tall (about 5 ft, 4 in) and weighing 144.6 kg (about 319 lb) is highlighted. (b) A scatterplot showing height versus BMI from the 500 individuals in the sample from NHANES. The same individual highlighted in (a) is marked here, with BMI 53.83. Fitted regression lines are shown in red.}\label{fig:unnamed-chunk-1}
\end{figure}


\end{knitrout}
	
\end{frame}










\begin{frame}[fragile]{Anscombe's quartet\footnote{\tiny{Anscombe, Francis J. (1973). Graphs in statistical analysis. The American Statistician, 27, 17–21. doi: 10.2307/2682899.}}}
	
\begin{knitrout}\scriptsize
\definecolor{shadecolor}{rgb}{0.969, 0.969, 0.969}\color{fgcolor}\begin{kframe}
\begin{alltt}
\hlkwd{library}\hlstd{(datasets)}
\hlkwd{data}\hlstd{(}\hlstr{"anscombe"}\hlstd{)}
\end{alltt}
\end{kframe}\begin{figure}

{\centering \includegraphics[width=0.7\linewidth]{figure/unnamed-chunk-2-1} 

}

\caption[All four panels have the exact same linear correlation coefficient]{All four panels have the exact same linear correlation coefficient}\label{fig:unnamed-chunk-2}
\end{figure}


\end{knitrout}


\end{frame}
						
						
						
						
\section{Two categorical variables}						
						\begin{frame}[fragile]{Two categorical variables}
							\protect\hypertarget{two-categorical-variables}{}
							
							A contingency table summarizes data for two categorical variables.
							
							\scriptsize
							
\begin{knitrout}\scriptsize
\definecolor{shadecolor}{rgb}{0.969, 0.969, 0.969}\color{fgcolor}\begin{kframe}
\begin{alltt}
\hlkwd{addmargins}\hlstd{(}\hlkwd{table}\hlstd{(famuss}\hlopt{$}\hlstd{race, famuss}\hlopt{$}\hlstd{actn3.r577x))}
\end{alltt}
\begin{verbatim}
##             
##               CC  CT  TT Sum
##   African Am  16   6   5  27
##   Asian       21  18  16  55
##   Caucasian  125 216 126 467
##   Hispanic     4  10   9  23
##   Other        7  11   5  23
##   Sum        173 261 161 595
\end{verbatim}
\end{kframe}
\end{knitrout}
							
						
							
							\normalsize
							
						\end{frame}
						
						\begin{frame}[fragile]{Two categorical variables \dots}
							\protect\hypertarget{two-categorical-variables-1}{}
							
							\scriptsize
							
							\scriptsize
							
\begin{knitrout}\scriptsize
\definecolor{shadecolor}{rgb}{0.969, 0.969, 0.969}\color{fgcolor}\begin{kframe}
\begin{alltt}
\hlcom{#row proportions}
\hlkwd{addmargins}\hlstd{(}\hlkwd{prop.table}\hlstd{(}\hlkwd{table}\hlstd{(famuss}\hlopt{$}\hlstd{race, famuss}\hlopt{$}\hlstd{actn3.r577x),} \hlnum{1}\hlstd{))}
\end{alltt}
\begin{verbatim}
##             
##                     CC        CT        TT       Sum
##   African Am 0.5925926 0.2222222 0.1851852 1.0000000
##   Asian      0.3818182 0.3272727 0.2909091 1.0000000
##   Caucasian  0.2676660 0.4625268 0.2698073 1.0000000
##   Hispanic   0.1739130 0.4347826 0.3913043 1.0000000
##   Other      0.3043478 0.4782609 0.2173913 1.0000000
##   Sum        1.7203376 1.9250652 1.3545972 5.0000000
\end{verbatim}
\end{kframe}
\end{knitrout}
							
\begin{knitrout}\scriptsize
\definecolor{shadecolor}{rgb}{0.969, 0.969, 0.969}\color{fgcolor}\begin{kframe}
\begin{alltt}
\hlcom{#column proportions}
\hlkwd{addmargins}\hlstd{(}\hlkwd{prop.table}\hlstd{(}\hlkwd{table}\hlstd{(famuss}\hlopt{$}\hlstd{race, famuss}\hlopt{$}\hlstd{actn3.r577x),} \hlnum{2}\hlstd{))}
\end{alltt}
\begin{verbatim}
##             
##                      CC         CT         TT        Sum
##   African Am 0.09248555 0.02298851 0.03105590 0.14652996
##   Asian      0.12138728 0.06896552 0.09937888 0.28973168
##   Caucasian  0.72254335 0.82758621 0.78260870 2.33273826
##   Hispanic   0.02312139 0.03831418 0.05590062 0.11733618
##   Other      0.04046243 0.04214559 0.03105590 0.11366392
##   Sum        1.00000000 1.00000000 1.00000000 3.00000000
\end{verbatim}
\end{kframe}
\end{knitrout}
							
							
							\normalsize
							
						\end{frame}
						
						\begin{frame}{Two categorical variables \dots}
							\protect\hypertarget{two-categorical-variables-2}{}
							
							\begin{figure}
								\centering
								\includegraphics[scale = 0.5]{figures/famussSegBarA.pdf}
								\caption{alt text}
							\end{figure}
							
							\emph{OI Biostat} Figure 1.35a, segmented bar plot
							
						\end{frame}
						
						\begin{frame}{Two categorical variables \dots}
							\protect\hypertarget{two-categorical-variables-3}{}
							
							\begin{figure}
								\centering
								\includegraphics[scale = 0.5]{figures/famussSegBarStaA.pdf}
								\caption{standardized segmented barplots}
							\end{figure}
							
							\emph{OI Biostat} Figure 1.35b, standardized segmented bar plot
							
						\end{frame}
					
					
					\begin{frame}[fragile]{Two categorical variables}
						
\begin{knitrout}\tiny
\definecolor{shadecolor}{rgb}{0.969, 0.969, 0.969}\color{fgcolor}

{\centering \includegraphics[width=\maxwidth]{figure/unnamed-chunk-3-1} 

}



\end{knitrout}
						
					\end{frame}
					
						
						\begin{frame}{Two categorical variables \dots}
							\protect\hypertarget{two-categorical-variables-4}{}
							
							\emph{Relative risk} (RR) is one way of summarizing data presented in a
							two-way table of study outcome by participant group.
							
							More in Lab 1 \dots
							
						\end{frame}
						
						\begin{frame}{A numerical variable and a categorical variable}
							\protect\hypertarget{a-numerical-variable-and-a-categorical-variable}{}
							
							\emph{FAMuSS} was designed to study the relationship between genotype at
							the location \emph{r577x} in the gene \emph{ACTN3} and muscle strength.
							
							Muscle strength was assessed by the percent change in non-dominant arm
							strength after resistance training (\texttt{ndrm.ch}).
							
							What visualization would be a good choice to make this comparison?
							
						\end{frame}
						
						\begin{frame}[fragile]{A numerical variable and a categorical variable
								\dots}
							\protect\hypertarget{a-numerical-variable-and-a-categorical-variable-1}{}
							
							\scriptsize
							
							\scriptsize
							
\begin{knitrout}\scriptsize
\definecolor{shadecolor}{rgb}{0.969, 0.969, 0.969}\color{fgcolor}\begin{kframe}
\begin{alltt}
\hlkwd{boxplot}\hlstd{(famuss}\hlopt{$}\hlstd{ndrm.ch} \hlopt{~} \hlstd{famuss}\hlopt{$}\hlstd{actn3.r577x)}
\end{alltt}
\end{kframe}

{\centering \includegraphics[width=\maxwidth]{figure/box-1-1} 

}



\end{knitrout}
							
							
							\normalsize
							
						\end{frame}
						
						\hypertarget{case-study-molecular-cancer-classification}{%
							\section{Case study: molecular cancer
								classification}\label{case-study-molecular-cancer-classification}}
						
						\begin{frame}{The potential value of genomic data in cancer}
							\protect\hypertarget{the-potential-value-of-genomic-data-in-cancer}{}
							
							The majority of cancers are diagnosed by an expert pathologist examining
							slides of malignant cells.
							
							Can that be done more accurately by characterizing the genetic makeup of
							the malignancy?
							
							\begin{itemize}
								\tightlist
								\item
								This is perhaps the major potential of genomic characterizations of
								tumors.
							\end{itemize}
							
							There are many forms of childhood leukemia.
							
							\begin{itemize}
								\item
								Acute myeloblastic leukemia (AML) and acute lymphoblastic leukemia
								(ALL) are the most common.
								\item
								AML is a cancer of the bone marrow, where white blood cells
								(lymphocytes) are produced.
								\item
								ALL is a cancer of the lymphocytes and is designated as B-cell (ALLB)
								or T-cell (ALLT).
							\end{itemize}
							
						\end{frame}
						
						\begin{frame}{Prognosis of the two cancers}
							\protect\hypertarget{prognosis-of-the-two-cancers}{}
							
							The probability that a child diagnosed with ALL is survives at least 5
							years after the diagnosis is approximately 90\%.
							
							Approximately 65\% of children diagnosed with AML survive at least 5
							years.
							
							The diagnosis of leukemia type determines the therapy that will be given
							to the child, and the successful treatments for ALL and AML are
							different.
							
							In 1999, Todd Golub from the Dana-Farber and the Broad Institute
							examined the possibility of classifying leukemia through using a genetic
							analysis of a blood sample.
							
						\end{frame}
						
						\begin{frame}{Analyzing the Golub data}
							\protect\hypertarget{analyzing-the-golub-data}{}
							
							We can re-analyze the Golub data using tools from graphical and
							numerical summaries.
							
							Our analysis will not be identical to the Golub analysis, but will be
							similar in spirit.
							
							The tools are straighforward\ldots   
							
							\begin{itemize}
								\item
								Thinking through the problem and assembling the tools is the hard
								part.
								\item
								The process is more important than the final recipe.
							\end{itemize}
							
						\end{frame}
						
						\begin{frame}{Gene expression (details in \emph{OI Biostat})}
							\protect\hypertarget{gene-expression-details-in-oi-biostat}{}
							
							\small
							
							\begin{itemize}
								\item
								The genetic code stored in DNA contains the information for producing
								the proteins that determine an organism's phenotype.
								\item
								Genes that are transcriptionally active (i.e.~turned ``on'') are
								transcribed into messenger RNA (mRNA) that gets translated into
								proteins.
								\item
								Genes can be switched on or off, and expressed at varying levels.
								Variations in gene expression produce the range of physical,
								biochemical, and developmental differences in cells and tissues.
								\item
								Quantifying the amount of RNA produced in a cell allows for a measure
								of gene expression.
								\item
								The transcriptome, or expression profile, is the complete set of RNA
								transcripts produced by the genome in a cell or set of cells.
							\end{itemize}
							
						\end{frame}
						
						\begin{frame}{Microarrays (details in \emph{OI Biostat})}
							\protect\hypertarget{microarrays-details-in-oi-biostat}{}
							
							\small
							
							\begin{itemize}
								\item
								Microarray technology is based on hybridization between two DNA
								strands, in which complementary nucleotide sequences specifically pair
								together.
								\item
								The mRNA from a sample is converted into complementary-DNA (cDNA),
								labeled with a fluorescent dye, and added to the microarray.
								\item
								When cDNA from the sample encounters complementary DNA probes, the two
								strands will hybridize, allowing the cDNA to adhere to specific spots
								on the slide.
								\item
								When the chip is illuminated and scanned, the intensity of
								fluorescence detected at each spot corresponds to the amount of bound
								cDNA.
								\item
								DNA microarrays do not directly quantify gene expression levels or
								quantity of mRNA present in a sample.
								\item
								The fluorescence intensity data only provide a relative measure of
								gene expression, showing which genes on the chip seem to be more or
								less active in relation to each other.
							\end{itemize}
							
						\end{frame}
						
						\begin{frame}{Microarrays}
							\protect\hypertarget{microarrays}{}
							
							\begin{figure}
								\centering
								\includegraphics{figures/microarray_schematic.jpg}
								\caption{fluorescence detection}
							\end{figure}
							
						\end{frame}
						
						\begin{frame}{The Golub clinical data}
							\protect\hypertarget{the-golub-clinical-data}{}
							
							Demographic variables described in \emph{OI Biostat} Table 1.54:
							
							\scriptsize
							
							\begin{longtable}[]{@{}ll@{}}
								\toprule
								\begin{minipage}[b]{0.12\columnwidth}\raggedright
									Variable\strut
								\end{minipage} & \begin{minipage}[b]{0.82\columnwidth}\raggedright
									Description\strut
								\end{minipage}\tabularnewline
								\midrule
								\endhead
								\begin{minipage}[t]{0.12\columnwidth}\raggedright
									Samples\strut
								\end{minipage} & \begin{minipage}[t]{0.82\columnwidth}\raggedright
									Sample or chip number. The material from each patient was examined on a
									separate chip and experimental run.\strut
								\end{minipage}\tabularnewline
								\begin{minipage}[t]{0.12\columnwidth}\raggedright
									BM.PB\strut
								\end{minipage} & \begin{minipage}[t]{0.82\columnwidth}\raggedright
									Type of patient material. BM denotes bone marrow; PB denotes a
									peripheral blood sample.\strut
								\end{minipage}\tabularnewline
								\begin{minipage}[t]{0.12\columnwidth}\raggedright
									Gender\strut
								\end{minipage} & \begin{minipage}[t]{0.82\columnwidth}\raggedright
									F for female, M for male.\strut
								\end{minipage}\tabularnewline
								\begin{minipage}[t]{0.12\columnwidth}\raggedright
									Source\strut
								\end{minipage} & \begin{minipage}[t]{0.82\columnwidth}\raggedright
									Hospital where the patient was treated.\strut
								\end{minipage}\tabularnewline
								\begin{minipage}[t]{0.12\columnwidth}\raggedright
									tissue.mf\strut
								\end{minipage} & \begin{minipage}[t]{0.82\columnwidth}\raggedright
									A variable showing the combination of type of patient material and sex
									of the patient. BM:f denotes bone marrow from a female patient,
									etc.\strut
								\end{minipage}\tabularnewline
								\begin{minipage}[t]{0.12\columnwidth}\raggedright
									cancer\strut
								\end{minipage} & \begin{minipage}[t]{0.82\columnwidth}\raggedright
									The type of leukemia; aml is acute myeloblastic leukemia, allB is acute
									lymphoblastic leukemia which started in B-cells (cells that mature into
									plasma cells) origin, and allT is acute lymphoblastic leukemia with
									T-cell origin (T-cells are a type of white blood cell).\strut
								\end{minipage}\tabularnewline
								\bottomrule
							\end{longtable}
							
						\end{frame}
						
						\begin{frame}{The Golub expression data}
							\protect\hypertarget{the-golub-expression-data}{}
							
							The expression data is contained in the last 7,129 columns.
							
							Each column is a variable with a name corresponding to the name of the
							probe on the microarray.
							
							The expression levels record fluorescence intensity for each gene.
							
							\begin{itemize}
								\item
								The intensity levels have no inherent biological meaning.
								\item
								Data have been normalized to adjust for variability between the
								separate arrays used for each patient.
							\end{itemize}
							
						\end{frame}
						
						\begin{frame}{Selected variables and columns from Golub data}
							\protect\hypertarget{selected-variables-and-columns-from-golub-data}{}
							
							\captionsetup[table]{labelformat=empty}
							\scriptsize
							
							\scriptsize
							
							\begin{longtable}[]{@{}rllrrr@{}}
								\caption{\emph{OI Biostat} Table 1.40}\tabularnewline
								\toprule
								Samples & Gender & cancer & AFFX-BioB-5\_at & AFFX-BioB-M\_at &
								AFFX-BioB-3\_at\tabularnewline
								\midrule
								\endfirsthead
								\toprule
								Samples & Gender & cancer & AFFX-BioB-5\_at & AFFX-BioB-M\_at &
								AFFX-BioB-3\_at\tabularnewline
								\midrule
								\endhead
								39 & F & allB & -1363.28 & -1058.59 & -541.47\tabularnewline
								40 & F & allB & -796.29 & -1167.10 & 7.54\tabularnewline
								42 & F & allB & -679.14 & -1069.83 & -690.30\tabularnewline
								47 & M & allB & -1164.40 & -1109.94 & -990.13\tabularnewline
								48 & F & allB & -1299.65 & -1402.00 & -1077.54\tabularnewline
								\bottomrule
							\end{longtable}
							
							\normalsize
							
						\end{frame}
						
						\begin{frame}{Analyzing the Golub leukemia data}
							\protect\hypertarget{analyzing-the-golub-leukemia-data}{}
							
							We will do an analysis in class using some of the simple but
							surprisingly powerful ideas behind numerical and graphical summaries.
							
							The goal of the Golub study was to develop a procedure for
							distinguishing between AML and ALL based only on the gene expression
							levels of a patient. There are two major issues to be addressed:
							
							\begin{enumerate}
								\item
								Which genes are the most informative for making a prediction?
								\item
								What is a workable strategy for predicting leukemia type from
								expression data for a specific set of genes?
							\end{enumerate}
							
						\end{frame}
						
						\begin{frame}[fragile]{Starting small\ldots{}}
							\protect\hypertarget{starting-small}{}
							
							\footnotesize
							
							\scriptsize
							
							\begin{verbatim}
							##    cancer        A         B        C        D
							## 69   allB 39307.96 35232.401 41170.76 35792.79
							## 67   allT 32281.88 41432.024 59328.51 49608.14
							## 55   allB 47429.94 35568.928 56074.96 42857.78
							## 56   allB 25533.87 16983.749 28056.75 32693.92
							## 59   allB 35960.55 24191.746 27637.90 22240.75
							## 52    aml 46177.95  6189.465 12557.24 34485.41
							## 53    aml 43790.70 33661.825 38380.30 29758.25
							## 51    aml 53420.05 26109.245 31427.20 23809.70
							## 50    aml 41241.59 37589.773 47325.77 30099.36
							## 54    aml 41300.57 49198.412 66026.10 56248.62
							\end{verbatim}
							
							\normalsize
							
						\end{frame}
						
\end{document}






\begin{frame}[fragile]{What levels of immunity are found in patients who have recovered from  COVID-19?\footnote{\tiny{Data were (imperfectly) scraped from the Postscript file ``behind'' the pdf file by Dr. Hanley}} }

	
\begin{knitrout}\tiny
\definecolor{shadecolor}{rgb}{0.969, 0.969, 0.969}\color{fgcolor}\begin{kframe}
\begin{alltt}
\hlstd{path} \hlkwb{<-}
  \hlstr{"http://www.biostat.mcgill.ca/hanley/statbook/immunogenicityChAdOx1.nCoV-19vaccine.txt"}
\hlstd{ds} \hlkwb{<-} \hlkwd{read.table}\hlstd{(path)}
\end{alltt}


{\ttfamily\noindent\bfseries\color{errorcolor}{\#\# Error in file(file, "{}rt"{}): cannot open the connection to 'http://www.biostat.mcgill.ca/hanley/statbook/immunogenicityChAdOx1.nCoV-19vaccine.txt'}}\begin{alltt}
\hlkwd{head}\hlstd{(ds)}
\end{alltt}


{\ttfamily\noindent\bfseries\color{errorcolor}{\#\# Error in head(ds): object 'ds' not found}}\begin{alltt}
\hlkwd{str}\hlstd{(ds)}
\end{alltt}


{\ttfamily\noindent\bfseries\color{errorcolor}{\#\# Error in str(ds): object 'ds' not found}}\begin{alltt}
\hlkwd{levels}\hlstd{(ds}\hlopt{$}\hlstd{RefIndexCategory)}
\end{alltt}


{\ttfamily\noindent\bfseries\color{errorcolor}{\#\# Error in levels(ds\$RefIndexCategory): object 'ds' not found}}\end{kframe}
\end{knitrout}

\end{frame}



\begin{frame}[fragile]{What levels of immunity are found in patients who have recovered from  COVID-19?}


\begin{knitrout}\tiny
\definecolor{shadecolor}{rgb}{0.969, 0.969, 0.969}\color{fgcolor}\begin{kframe}
\begin{alltt}
\hlstd{natural} \hlkwb{<-} \hlstd{ds[ds}\hlopt{$}\hlstd{RefIndexCategory}\hlopt{==}\hlstr{"Convalescent"}\hlstd{,]}
\end{alltt}


{\ttfamily\noindent\bfseries\color{errorcolor}{\#\# Error in eval(expr, envir, enclos): object 'ds' not found}}\begin{alltt}
\hlkwd{hist}\hlstd{(natural}\hlopt{$}\hlstd{IgGResponse.log10.ElisaUnits,}
     \hlkwc{breaks} \hlstd{=} \hlnum{20}\hlstd{,} \hlkwc{col} \hlstd{=} \hlstr{"lightblue"}\hlstd{)}
\end{alltt}


{\ttfamily\noindent\bfseries\color{errorcolor}{\#\# Error in hist(natural\$IgGResponse.log10.ElisaUnits, breaks = 20, col = "{}lightblue"{}): object 'natural' not found}}\end{kframe}
\end{knitrout}

\end{frame}


\begin{frame}[fragile]{Three different methods of calculating the mean}

\begin{figure}
\begin{minipage}[h]{0.40\linewidth}
\begin{knitrout}\tiny
\definecolor{shadecolor}{rgb}{0.969, 0.969, 0.969}\color{fgcolor}\begin{kframe}
\begin{alltt}
\hlkwd{summary}\hlstd{(natural}\hlopt{$}\hlstd{IgGResponse.log10.ElisaUnits)}
\end{alltt}


{\ttfamily\noindent\bfseries\color{errorcolor}{\#\# Error in summary(natural\$IgGResponse.log10.ElisaUnits): object 'natural' not found}}\begin{alltt}
\hlkwd{boxplot}\hlstd{(natural}\hlopt{$}\hlstd{IgGResponse.log10.ElisaUnits,}
        \hlkwc{col} \hlstd{=} \hlstr{"lightblue"}\hlstd{,}
        \hlkwc{ylab} \hlstd{=} \hlstr{"Immunoglobulin G (IgG) response"}\hlstd{)}
\end{alltt}


{\ttfamily\noindent\bfseries\color{errorcolor}{\#\# Error in boxplot(natural\$IgGResponse.log10.ElisaUnits, col = "{}lightblue"{}, : object 'natural' not found}}\begin{alltt}
\hlkwd{grid}\hlstd{(}\hlkwc{lty} \hlstd{=} \hlstr{"dashed"}\hlstd{)}
\end{alltt}


{\ttfamily\noindent\bfseries\color{errorcolor}{\#\# Error in int\_abline(a = a, b = b, h = h, v = v, untf = untf, ...): plot.new has not been called yet}}\end{kframe}
\end{knitrout}

\end{minipage}
\hspace{0.4cm}
\begin{minipage}[h]{0.50\linewidth}
\begin{knitrout}\tiny
\definecolor{shadecolor}{rgb}{0.969, 0.969, 0.969}\color{fgcolor}\begin{kframe}
\begin{alltt}
\hlkwd{t.test}\hlstd{(natural}\hlopt{$}\hlstd{IgGResponse.log10.ElisaUnits)}
\end{alltt}


{\ttfamily\noindent\bfseries\color{errorcolor}{\#\# Error in t.test(natural\$IgGResponse.log10.ElisaUnits): object 'natural' not found}}\begin{alltt}
\hlstd{fit1} \hlkwb{<-} \hlkwd{glm}\hlstd{(IgGResponse.log10.ElisaUnits} \hlopt{~} \hlnum{1}\hlstd{,} \hlkwc{data} \hlstd{= natural)}
\end{alltt}


{\ttfamily\noindent\bfseries\color{errorcolor}{\#\# Error in is.data.frame(data): object 'natural' not found}}\begin{alltt}
\hlkwd{summary}\hlstd{(fit1)}
\end{alltt}


{\ttfamily\noindent\bfseries\color{errorcolor}{\#\# Error in summary(fit1): object 'fit1' not found}}\begin{alltt}
\hlkwd{confint}\hlstd{(fit1)}
\end{alltt}


{\ttfamily\noindent\bfseries\color{errorcolor}{\#\# Error in confint(fit1): object 'fit1' not found}}\end{kframe}
\end{knitrout}
\end{minipage}
\end{figure}


\end{frame}



\begin{frame}[fragile]{Naturally vs. vaccine-induced response levels}
	
	
\begin{knitrout}\tiny
\definecolor{shadecolor}{rgb}{0.969, 0.969, 0.969}\color{fgcolor}\begin{kframe}
\begin{alltt}
\hlstd{p1} \hlkwb{<-} \hlkwd{ggplot}\hlstd{(}\hlkwc{data} \hlstd{= ds,} \hlkwc{mapping} \hlstd{=} \hlkwd{aes}\hlstd{(}\hlkwc{x} \hlstd{= RefIndexCategory,} \hlkwc{y} \hlstd{= IgGResponse.log10.ElisaUnits,}
    \hlkwc{fill} \hlstd{= RefIndexCategory))} \hlopt{+} \hlkwd{geom_jitter}\hlstd{(}\hlkwc{alpha} \hlstd{=} \hlnum{0.3}\hlstd{)} \hlopt{+} \hlkwd{theme_minimal}\hlstd{()} \hlopt{+} \hlkwd{theme}\hlstd{(}\hlkwc{legend.position} \hlstd{=} \hlstr{"none"}\hlstd{)}
\end{alltt}


{\ttfamily\noindent\bfseries\color{errorcolor}{\#\# Error in ggplot(data = ds, mapping = aes(x = RefIndexCategory, y = IgGResponse.log10.ElisaUnits, : object 'ds' not found}}\begin{alltt}
\hlstd{p1} \hlopt{+} \hlkwd{geom_violin}\hlstd{()}
\end{alltt}


{\ttfamily\noindent\bfseries\color{errorcolor}{\#\# Error in eval(expr, envir, enclos): object 'p1' not found}}\begin{alltt}
\hlstd{p1} \hlopt{+} \hlkwd{geom_boxplot}\hlstd{()}
\end{alltt}


{\ttfamily\noindent\bfseries\color{errorcolor}{\#\# Error in eval(expr, envir, enclos): object 'p1' not found}}\end{kframe}
\end{knitrout}
	
\end{frame}



\begin{frame}[fragile]{Comparing means using classic methods}

\textbf{1. Numerical summary} \\

\begin{knitrout}\tiny
\definecolor{shadecolor}{rgb}{0.969, 0.969, 0.969}\color{fgcolor}\begin{kframe}
\begin{alltt}
\hlkwd{by}\hlstd{(ds}\hlopt{$}\hlstd{IgGResponse.log10.ElisaUnits,ds}\hlopt{$}\hlstd{RefIndexCategory,summary)}
\end{alltt}


{\ttfamily\noindent\bfseries\color{errorcolor}{\#\# Error in by(ds\$IgGResponse.log10.ElisaUnits, ds\$RefIndexCategory, summary): object 'ds' not found}}\end{kframe}
\end{knitrout}
\pause

\vspace*{0.3in}

\textbf{2. Another ``dot'' test} \\

\begin{knitrout}\tiny
\definecolor{shadecolor}{rgb}{0.969, 0.969, 0.969}\color{fgcolor}\begin{kframe}
\begin{alltt}
\hlkwd{t.test}\hlstd{(IgGResponse.log10.ElisaUnits} \hlopt{~} \hlstd{RefIndexCategory,} \hlkwc{data} \hlstd{= ds)}
\end{alltt}


{\ttfamily\noindent\bfseries\color{errorcolor}{\#\# Error in eval(m\$data, parent.frame()): object 'ds' not found}}\end{kframe}
\end{knitrout}

\end{frame}



\begin{frame}[fragile]{Comparing means using regression}
	
\textbf{3. Regression} \\
	
\begin{knitrout}\tiny
\definecolor{shadecolor}{rgb}{0.969, 0.969, 0.969}\color{fgcolor}\begin{kframe}
\begin{alltt}
\hlstd{fit2} \hlkwb{<-} \hlkwd{glm}\hlstd{(IgGResponse.log10.ElisaUnits} \hlopt{~} \hlstd{RefIndexCategory,} \hlkwc{data} \hlstd{= ds)}
\end{alltt}


{\ttfamily\noindent\bfseries\color{errorcolor}{\#\# Error in is.data.frame(data): object 'ds' not found}}\begin{alltt}
\hlkwd{print}\hlstd{(}\hlkwd{summary}\hlstd{(fit2),} \hlkwc{signif.star} \hlstd{=} \hlnum{FALSE}\hlstd{)}
\end{alltt}


{\ttfamily\noindent\bfseries\color{errorcolor}{\#\# Error in summary(fit2): object 'fit2' not found}}\begin{alltt}
\hlkwd{confint}\hlstd{(fit2)}
\end{alltt}


{\ttfamily\noindent\bfseries\color{errorcolor}{\#\# Error in confint(fit2): object 'fit2' not found}}\end{kframe}
\end{knitrout}

\end{frame}


\begin{frame}[fragile]{Fitted regression line}
	
\begin{knitrout}\tiny
\definecolor{shadecolor}{rgb}{0.969, 0.969, 0.969}\color{fgcolor}\begin{kframe}
\begin{alltt}
\hlkwd{plot}\hlstd{(ds}\hlopt{$}\hlstd{RefIndexCategory, ds}\hlopt{$}\hlstd{IgGResponse.log10.ElisaUnits,} \hlkwc{pch}\hlstd{=}\hlnum{19}\hlstd{,} \hlkwc{cex}\hlstd{=}\hlnum{0.5}\hlstd{)}
\end{alltt}


{\ttfamily\noindent\bfseries\color{errorcolor}{\#\# Error in plot(ds\$RefIndexCategory, ds\$IgGResponse.log10.ElisaUnits, pch = 19, : object 'ds' not found}}\begin{alltt}
\hlkwd{abline}\hlstd{(}\hlkwc{h} \hlstd{=} \hlkwd{seq}\hlstd{(}\hlnum{0}\hlstd{,}\hlnum{4}\hlstd{,}\hlnum{0.5}\hlstd{),}\hlkwc{col} \hlstd{=} \hlstr{"lightblue"}\hlstd{)}
\end{alltt}


{\ttfamily\noindent\bfseries\color{errorcolor}{\#\# Error in int\_abline(a = a, b = b, h = h, v = v, untf = untf, ...): plot.new has not been called yet}}\begin{alltt}
\hlkwd{lines}\hlstd{(ds}\hlopt{$}\hlstd{RefIndexCategory, fit2}\hlopt{$}\hlstd{fitted.values,} \hlkwc{col} \hlstd{=} \hlstr{"red"}\hlstd{,} \hlkwc{lwd} \hlstd{=} \hlnum{3}\hlstd{)}
\end{alltt}


{\ttfamily\noindent\bfseries\color{errorcolor}{\#\# Error in lines(ds\$RefIndexCategory, fit2\$fitted.values, col = "{}red"{}, lwd = 3): object 'ds' not found}}\end{kframe}
\end{knitrout}
	
\end{frame}




\section{Case study 2: Comparison of Estimated Rates of Coronavirus Disease 2019 (COVID-19) in Border Counties in Iowa Without a Stay-at-Home Order and Border Counties in Illinois With a Stay-at-Home Order}

\begin{frame}{Comparing Iowa and Illinois Cases\footnote{\tiny\url{https://jamanetwork.com/journals/jamanetworkopen/fullarticle/2766229}}}
	\centering
	\includegraphics[scale=0.25]{002-masks.png}	
\end{frame}


\begin{frame}[fragile]{Are the difference in curves real? Or just random variation?}

\begin{itemize}
	\item This study compared COVID-19 cases in border counties in \textcolor{red}{Iowa, which did not issue a stay-at-home order}, with cases in border counties in \textcolor{blue}{Illinois, which did issue a stay-at-home order}.
\end{itemize}

\begin{knitrout}\tiny
\definecolor{shadecolor}{rgb}{0.969, 0.969, 0.969}\color{fgcolor}\begin{kframe}


{\ttfamily\noindent\bfseries\color{errorcolor}{\#\# Error in download.file(path, destfile = file\_path): cannot open URL 'http://www.biostat.mcgill.ca/hanley/statbook/IowaIllinoisBorderFig1.png'}}\end{kframe}
\end{knitrout}

\end{frame}





\begin{frame}[fragile]{Freely available county level data from NYTimes\footnote{\tiny{\url{https://github.com/nytimes/covid-19-data}}}}
	
\begin{knitrout}\tiny
\definecolor{shadecolor}{rgb}{0.969, 0.969, 0.969}\color{fgcolor}\begin{kframe}
\begin{alltt}
\hlkwd{library}\hlstd{(covdata)} \hlcom{# remotes::install_github("kjhealy/covdata")}
\hlkwd{library}\hlstd{(dplyr);} \hlkwd{library}\hlstd{(tidyr);} \hlkwd{library}\hlstd{(ggplot2);} \hlkwd{library}\hlstd{(readr)}

\hlcom{# get population data from https://covid19.census.gov/datasets/}
\hlstd{pop_county} \hlkwb{<-} \hlkwd{read_csv}\hlstd{(}\hlstr{"https://opendata.arcgis.com/datasets/21843f238cbb46b08615fc53e19e0daf_1.csv"}\hlstd{)} \hlopt
              \hlstd{dplyr}\hlopt{::}\hlkwd{rename}\hlstd{(}\hlkwc{fips} \hlstd{= GEOID,} \hlkwc{population} \hlstd{= B01001_001E,} \hlkwc{state} \hlstd{= State)} \hlopt
              \hlstd{dplyr}\hlopt{::}\hlkwd{select}\hlstd{(state, fips, population)}
\end{alltt}


{\ttfamily\noindent\bfseries\color{errorcolor}{\#\# Error in open.connection(con, "{}rb"{}): Could not resolve host: opendata.arcgis.com}}\begin{alltt}
\hlstd{county_level} \hlkwb{<-} \hlstd{nytcovcounty} \hlopt
                \hlstd{dplyr}\hlopt{::}\hlkwd{left_join}\hlstd{(pop_county,} \hlkwc{by} \hlstd{=} \hlkwd{c}\hlstd{(}\hlstr{"state"}\hlstd{,}\hlstr{"fips"}\hlstd{))} \hlopt
                \hlstd{dplyr}\hlopt{::}\hlkwd{mutate}\hlstd{(}\hlkwc{cases.per.10k} \hlstd{= cases}\hlopt{/}\hlstd{population} \hlopt{*} \hlnum{1e4}\hlstd{)} \hlopt
                \hlstd{dplyr}\hlopt{::}\hlkwd{filter}\hlstd{(state} \hlopt \hlkwd{c}\hlstd{(}\hlstr{"Iowa"}\hlstd{,}\hlstr{"Illinois"}\hlstd{))} \hlopt
                \hlstd{dplyr}\hlopt{::}\hlkwd{group_by}\hlstd{(county)}
\end{alltt}


{\ttfamily\noindent\bfseries\color{errorcolor}{\#\# Error in is.data.frame(y): object 'pop\_county' not found}}\begin{alltt}
\hlstd{pop_state} \hlkwb{<-} \hlstd{pop_county} \hlopt
             \hlstd{dplyr}\hlopt{::}\hlkwd{group_by}\hlstd{(state)} \hlopt
             \hlstd{dplyr}\hlopt{::}\hlkwd{summarise}\hlstd{(}\hlkwc{population} \hlstd{=} \hlkwd{sum}\hlstd{(population,} \hlkwc{na.rm} \hlstd{=} \hlnum{TRUE}\hlstd{))}
\end{alltt}


{\ttfamily\noindent\bfseries\color{errorcolor}{\#\# Error in eval(lhs, parent, parent): object 'pop\_county' not found}}\begin{alltt}
\hlstd{state_level} \hlkwb{<-} \hlstd{county_level} \hlopt
               \hlstd{dplyr}\hlopt{::}\hlkwd{group_by}\hlstd{(state, date)} \hlopt
               \hlstd{dplyr}\hlopt{::}\hlkwd{filter}\hlstd{(date} \hlopt{>=} \hlstr{"2020-03-15"}\hlstd{)} \hlopt
               \hlstd{dplyr}\hlopt{::}\hlkwd{summarise}\hlstd{(}\hlkwc{cases} \hlstd{=} \hlkwd{sum}\hlstd{(cases))} \hlopt
               \hlstd{dplyr}\hlopt{::}\hlkwd{left_join}\hlstd{(pop_state,} \hlkwc{by} \hlstd{=} \hlstr{"state"}\hlstd{)} \hlopt
               \hlstd{dplyr}\hlopt{::}\hlkwd{mutate}\hlstd{(}\hlkwc{cases.per.10k} \hlstd{= cases} \hlopt{/} \hlstd{population} \hlopt{*} \hlnum{1e4}\hlstd{,} \hlkwc{state} \hlstd{=} \hlkwd{factor}\hlstd{(state),}
                             \hlkwc{time} \hlstd{=} \hlkwd{as.numeric}\hlstd{(date} \hlopt{-} \hlkwd{min}\hlstd{(date))} \hlopt{+} \hlnum{1}\hlstd{)}
\end{alltt}


{\ttfamily\noindent\bfseries\color{errorcolor}{\#\# Error in eval(lhs, parent, parent): object 'county\_level' not found}}\begin{alltt}
\hlkwd{head}\hlstd{(state_level)}
\end{alltt}


{\ttfamily\noindent\bfseries\color{errorcolor}{\#\# Error in head(state\_level): object 'state\_level' not found}}\end{kframe}
\end{knitrout}
	
\end{frame}



\begin{frame}[fragile]{County level cases for Iowa and Illinois - log10 scale}
	
\begin{knitrout}\tiny
\definecolor{shadecolor}{rgb}{0.969, 0.969, 0.969}\color{fgcolor}\begin{kframe}
\begin{alltt}
\hlkwd{ggplot}\hlstd{(}\hlkwc{data} \hlstd{= county_level,} \hlkwc{mapping} \hlstd{=} \hlkwd{aes}\hlstd{(}\hlkwc{x} \hlstd{= date,} \hlkwc{y} \hlstd{= cases,} \hlkwc{group} \hlstd{= county))} \hlopt{+}
  \hlkwd{geom_line}\hlstd{(}\hlkwc{size} \hlstd{=} \hlnum{0.25}\hlstd{,} \hlkwc{color} \hlstd{=} \hlstr{"gray20"}\hlstd{)} \hlopt{+}
  \hlkwd{scale_x_date}\hlstd{(}\hlkwc{date_breaks} \hlstd{=} \hlstr{"1 month"}\hlstd{,} \hlkwc{date_labels} \hlstd{=} \hlstr{"%b"}\hlstd{)}\hlopt{+}
  \hlkwd{scale_y_log10}\hlstd{(}\hlkwc{labels} \hlstd{= scales}\hlopt{::}\hlkwd{label_number_si}\hlstd{())} \hlopt{+}
  \hlkwd{guides}\hlstd{(}\hlkwc{color} \hlstd{=} \hlnum{FALSE}\hlstd{)} \hlopt{+} \hlkwd{facet_wrap}\hlstd{(}\hlopt{~} \hlstd{state,} \hlkwc{ncol} \hlstd{=} \hlnum{2}\hlstd{)} \hlopt{+}
  \hlkwd{labs}\hlstd{(}\hlkwc{title} \hlstd{=} \hlstr{"COVID-19 Cases in Iowa and Illinois by County"}\hlstd{,}
       \hlkwc{x} \hlstd{=} \hlstr{"Date"}\hlstd{,} \hlkwc{y} \hlstd{=} \hlstr{"No. of cases (log10 scale)"}\hlstd{,} \hlkwc{caption} \hlstd{=} \hlstr{"Data: The New York Times"}\hlstd{)} \hlopt{+}
  \hlkwd{theme_minimal}\hlstd{()}
\end{alltt}


{\ttfamily\noindent\bfseries\color{errorcolor}{\#\# Error in ggplot(data = county\_level, mapping = aes(x = date, y = cases, : object 'county\_level' not found}}\end{kframe}
\end{knitrout}
	
\end{frame}




\begin{frame}[fragile]{County level cases for Iowa and Illinois - per capita}
	
\begin{knitrout}\tiny
\definecolor{shadecolor}{rgb}{0.969, 0.969, 0.969}\color{fgcolor}\begin{kframe}
\begin{alltt}
\hlkwd{ggplot}\hlstd{(}\hlkwc{data} \hlstd{= county_level,} \hlkwc{mapping} \hlstd{=} \hlkwd{aes}\hlstd{(}\hlkwc{x} \hlstd{= date,} \hlkwc{y} \hlstd{= cases.per.10k,} \hlkwc{group} \hlstd{= county))} \hlopt{+}
  \hlkwd{geom_line}\hlstd{(}\hlkwc{size} \hlstd{=} \hlnum{0.25}\hlstd{,} \hlkwc{color} \hlstd{=} \hlstr{"gray20"}\hlstd{)} \hlopt{+}
  \hlkwd{scale_x_date}\hlstd{(}\hlkwc{date_breaks} \hlstd{=} \hlstr{"1 month"}\hlstd{,} \hlkwc{date_labels} \hlstd{=} \hlstr{"%b"}\hlstd{)}\hlopt{+}
  \hlkwd{scale_y_continuous}\hlstd{(}\hlkwc{labels} \hlstd{= scales}\hlopt{::}\hlkwd{label_number_si}\hlstd{())} \hlopt{+}
  \hlkwd{guides}\hlstd{(}\hlkwc{color} \hlstd{=} \hlnum{FALSE}\hlstd{)} \hlopt{+} \hlkwd{facet_wrap}\hlstd{(}\hlopt{~} \hlstd{state,} \hlkwc{ncol} \hlstd{=} \hlnum{2}\hlstd{)} \hlopt{+}
  \hlkwd{labs}\hlstd{(}\hlkwc{title} \hlstd{=} \hlstr{"COVID-19 Cases in Iowa and Illinois by County"}\hlstd{,}
       \hlkwc{x} \hlstd{=} \hlstr{"Date"}\hlstd{,} \hlkwc{y} \hlstd{=} \hlstr{"No. of cases per 10 000"}\hlstd{,} \hlkwc{caption} \hlstd{=} \hlstr{"Data: The New York Times"}\hlstd{)} \hlopt{+}
  \hlkwd{theme_minimal}\hlstd{()}
\end{alltt}


{\ttfamily\noindent\bfseries\color{errorcolor}{\#\# Error in ggplot(data = county\_level, mapping = aes(x = date, y = cases.per.10k, : object 'county\_level' not found}}\end{kframe}
\end{knitrout}
	
\end{frame}



\begin{frame}[fragile]{State level cases for Iowa and Illinois - per capita}
	
\begin{knitrout}\tiny
\definecolor{shadecolor}{rgb}{0.969, 0.969, 0.969}\color{fgcolor}\begin{kframe}
\begin{alltt}
\hlkwd{ggplot}\hlstd{(}\hlkwc{data} \hlstd{= state_level,} \hlkwc{mapping} \hlstd{=} \hlkwd{aes}\hlstd{(}\hlkwc{x} \hlstd{= date,} \hlkwc{y} \hlstd{= cases.per.10k,} \hlkwc{color} \hlstd{= state))} \hlopt{+}
  \hlkwd{geom_line}\hlstd{(}\hlkwc{size} \hlstd{=} \hlnum{1}\hlstd{)} \hlopt{+}
  \hlkwd{scale_x_date}\hlstd{(}\hlkwc{date_breaks} \hlstd{=} \hlstr{"1 month"}\hlstd{,} \hlkwc{date_labels} \hlstd{=} \hlstr{"%b"}\hlstd{)}\hlopt{+}
  \hlkwd{scale_y_continuous}\hlstd{(}\hlkwc{labels} \hlstd{= scales}\hlopt{::}\hlkwd{label_number_si}\hlstd{())} \hlopt{+}
  \hlkwd{labs}\hlstd{(}\hlkwc{title} \hlstd{=} \hlstr{"COVID-19 Cases in Iowa and Illinois"}\hlstd{,}
       \hlkwc{subtitle} \hlstd{=} \hlstr{"Cases since March 15, 2020"}\hlstd{,}
       \hlkwc{x} \hlstd{=} \hlstr{"Date"}\hlstd{,} \hlkwc{y} \hlstd{=} \hlstr{"No. of cases per 10 000"}\hlstd{,} \hlkwc{caption} \hlstd{=} \hlstr{"Data: The New York Times"}\hlstd{)} \hlopt{+}
  \hlkwd{theme_minimal}\hlstd{()}
\end{alltt}


{\ttfamily\noindent\bfseries\color{errorcolor}{\#\# Error in ggplot(data = state\_level, mapping = aes(x = date, y = cases.per.10k, : object 'state\_level' not found}}\end{kframe}
\end{knitrout}
	
\end{frame}


\begin{frame}[fragile]{Are the findings in the paper reproducible?}
	
\begin{knitrout}\tiny
\definecolor{shadecolor}{rgb}{0.969, 0.969, 0.969}\color{fgcolor}\begin{kframe}
\begin{alltt}
\hlstd{fit3} \hlkwb{<-} \hlkwd{glm}\hlstd{(cases.per.10k} \hlopt{~} \hlstd{state}\hlopt{*}\hlstd{time,} \hlkwc{data} \hlstd{= state_level)}
\end{alltt}


{\ttfamily\noindent\bfseries\color{errorcolor}{\#\# Error in is.data.frame(data): object 'state\_level' not found}}\begin{alltt}
\hlkwd{summary}\hlstd{(fit3)}
\end{alltt}


{\ttfamily\noindent\bfseries\color{errorcolor}{\#\# Error in summary(fit3): object 'fit3' not found}}\end{kframe}
\end{knitrout}
	
\end{frame}


\begin{frame}[fragile]{Model-based predictions}
	
\begin{knitrout}\tiny
\definecolor{shadecolor}{rgb}{0.969, 0.969, 0.969}\color{fgcolor}\begin{kframe}
\begin{alltt}
\hlkwd{library}\hlstd{(ggeffects)}
\hlstd{ggeffects}\hlopt{::}\hlkwd{ggpredict}\hlstd{(fit3,} \hlkwc{terms} \hlstd{=} \hlstr{"state"}\hlstd{)} \hlopt
  \hlkwd{plot}\hlstd{()}
\end{alltt}


{\ttfamily\noindent\bfseries\color{errorcolor}{\#\# Error in is.gamm(model): object 'fit3' not found}}\end{kframe}
\end{knitrout}
	
\end{frame}


%\begin{frame}[allowframebreaks]
%\nocite{breiman1984classification}
%	\nocite{friedman2001elements}
%	\nocite{james2013introduction}
%	\nocite{lopez2015arbres}
%	\frametitle{References}
%\printbibliography
%\end{frame}


\begin{frame}[fragile]{Session Info}
	\tiny
	
\begin{knitrout}\tiny
\definecolor{shadecolor}{rgb}{0.969, 0.969, 0.969}\color{fgcolor}\begin{kframe}
\begin{verbatim}
R version 3.6.2 (2019-12-12)
Platform: x86_64-pc-linux-gnu (64-bit)
Running under: Pop!_OS 19.10

Matrix products: default
BLAS:   /usr/lib/x86_64-linux-gnu/openblas/libblas.so.3
LAPACK: /usr/lib/x86_64-linux-gnu/libopenblasp-r0.3.7.so

attached base packages:
[1] tools     stats     graphics  grDevices utils     datasets  methods  
[8] base     

other attached packages:
 [1] ggeffects_0.14.1    covdata_0.4.4       openintro_2.0.0    
 [4] usdata_0.1.0        cherryblossom_0.1.0 airports_0.1.0     
 [7] oibiostat_0.2.0     NCStats_0.4.7       FSA_0.8.30         
[10] forcats_0.5.0       stringr_1.4.0       dplyr_1.0.2        
[13] purrr_0.3.4         readr_1.3.1         tidyr_1.1.2        
[16] tibble_3.0.3        ggplot2_3.3.2.9000  tidyverse_1.3.0    
[19] knitr_1.29         

loaded via a namespace (and not attached):
 [1] Rcpp_1.0.4.6       lubridate_1.7.4    assertthat_0.2.1   digest_0.6.25     
 [5] R6_2.4.1           cellranger_1.1.0   plyr_1.8.6         backports_1.1.9   
 [9] reprex_0.3.0       evaluate_0.14      httr_1.4.1         highr_0.8         
[13] pillar_1.4.6       TeachingDemos_2.12 rlang_0.4.7        curl_4.3          
[17] readxl_1.3.1       rstudioapi_0.11    labeling_0.3       munsell_0.5.0     
[21] broom_0.7.0        compiler_3.6.2     modelr_0.1.5       xfun_0.16         
[25] pkgconfig_2.0.3    insight_0.8.1      tidyselect_1.1.0   fansi_0.4.1       
[29] crayon_1.3.4       dbplyr_1.4.2       withr_2.2.0        MASS_7.3-51.5     
[33] grid_3.6.2         jsonlite_1.7.0     gtable_0.3.0       lifecycle_0.2.0   
[37] DBI_1.1.0          magrittr_1.5       formatR_1.7        scales_1.1.1      
[41] cli_2.0.2          stringi_1.4.6      farver_2.0.3       fs_1.3.2          
[45] xml2_1.3.0         ellipsis_0.3.1     generics_0.0.2     vctrs_0.3.4       
[49] sjlabelled_1.1.3   glue_1.4.2         hms_0.5.3          colorspace_1.4-1  
[53] rvest_0.3.5        haven_2.3.1       
\end{verbatim}
\end{kframe}
\end{knitrout}
	
\end{frame}

\end{document}
