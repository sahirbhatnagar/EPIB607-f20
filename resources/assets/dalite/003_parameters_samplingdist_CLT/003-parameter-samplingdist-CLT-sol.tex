\documentclass[letterpaper,12pt,twoside,printwatermark=false]{pinp}

%% Some pieces required from the pandoc template
\providecommand{\tightlist}{%
  \setlength{\itemsep}{0pt}\setlength{\parskip}{0pt}}

% Use the lineno option to display guide line numbers if required.
% Note that the use of elements such as single-column equations
% may affect the guide line number alignment.

\usepackage[T1]{fontenc}
\usepackage[utf8]{inputenc}

% The geometry package layout settings need to be set here...
\geometry{layoutsize={0.95588\paperwidth,0.98864\paperheight},%
          layouthoffset=0.02206\paperwidth,%
		  layoutvoffset=0.00568\paperheight}

\definecolor{pinpblue}{HTML}{185FAF}  % imagecolorpicker on blue for new R logo
\definecolor{pnasbluetext}{RGB}{101,0,0} %



\title{DALITE Q3 - Parameters, Sampling Distributions and the Central Limit
Theorem Solutions.}

\author[a]{EPIB607 - Inferential Statistics}

  \affil[a]{Fall 2019, McGill University}

\setcounter{secnumdepth}{5}

% Please give the surname of the lead author for the running footer
\leadauthor{Bhatnagar and Hanley}

% Keywords are not mandatory, but authors are strongly encouraged to provide them. If provided, please include two to five keywords, separated by the pipe symbol, e.g:
 \keywords{  Parameters and statistics |  Sampling distributions |  Central Limit Theorem (CLT)  }  

\begin{abstract}
This DALITE quiz will cover the building blocks of statistical
inference.
\end{abstract}

\dates{This version was compiled on \today} 

% initially we use doi so keep for backwards compatibility
\doifooter{\url{https://sahirbhatnagar.com/EPIB607/}}
% new name is doi_footer

\pinpfootercontents{DALITE Q3 due Sepetember 24, 2019 by 5pm}

\begin{document}

% Optional adjustment to line up main text (after abstract) of first page with line numbers, when using both lineno and twocolumn options.
% You should only change this length when you've finalised the article contents.
\verticaladjustment{-2pt}

\maketitle
\thispagestyle{firststyle}
\ifthenelse{\boolean{shortarticle}}{\ifthenelse{\boolean{singlecolumn}}{\abscontentformatted}{\abscontent}}{}

% If your first paragraph (i.e. with the \dropcap) contains a list environment (quote, quotation, theorem, definition, enumerate, itemize...), the line after the list may have some extra indentation. If this is the case, add \parshape=0 to the end of the list environment.


\hypertarget{parameters-q1}{%
\section{Parameters Q1}\label{parameters-q1}}

Which of the following statements is false?

\begin{enumerate}
\def\labelenumi{\alph{enumi})}
\tightlist
\item
  A parameter is a constant of unknown magnitude in a statistical model
\item
  The value of a parameter is in general unknown
\item
  A statistic is a number derived from a sample
\item
  \textbf{The population standard deviation can be estimated from the
  sample provided that we have a sample size greater than 30 (Correct)}
\end{enumerate}

\hypertarget{correct-rationales}{%
\subsection{Correct rationales}\label{correct-rationales}}

\begin{itemize}
\tightlist
\item
  No, you can have a standard deviation in a sample that has a `n value'
  of smaller than 30 this would just change the estimated standard
  deviations accuracy.
\item
  Population standard deviation can be estimated from a sample size
  regardless of the specific number
\end{itemize}

\hypertarget{incorrect-rationales}{%
\subsection{Incorrect rationales}\label{incorrect-rationales}}

\begin{itemize}
\tightlist
\item
  Generally the sample standard deviation is estimated from the
  population standard deviation using the square root of the sample
  size.
\end{itemize}

\hypertarget{sampling-distributions-q1}{%
\section{Sampling Distributions Q1}\label{sampling-distributions-q1}}

A newborn baby has extermely low birth weight (ELBW) if it weighs less
than 1000 grams. A study of the health of such children in later years
examined a random sample of 219 children. Their mean weight at birth was
\(\bar{y}\) = 810 grams. This sample mean (\(\bar{y}\)) is an unbiased
estimator of the mean weight \(\mu\) in the population of all ELBW
babies. This means that

\begin{enumerate}
\def\labelenumi{\alph{enumi})}
\tightlist
\item
  \textbf{In many samples from this population, the mean of many values
  of \(\bar{y}\) will be equal to \(\mu\) (Correct)}
\item
  As we take larger and larger samples from this population, the sample
  mean \(\bar{y}\) will get closer and closer to \(\mu\)
\item
  In many samples from this population, the many values of the sample
  mean \(\bar{y}\) will have a distribution that is close to Normal
\end{enumerate}

\hypertarget{correct-rationales-1}{%
\subsection{Correct rationales}\label{correct-rationales-1}}

\begin{itemize}
\tightlist
\item
  As the sample mean \(\bar{y}\) is an unbiased estimator of the mean
  weight \(\mu\) in the population, the mean of many values of
  \(\bar{y}\) will be equal to \(\mu\). Further, if we plot the sample
  means, the resulting sampling distribution of \(\bar{y}\) will have
  the same mean value as the mean in the population distribution.
\item
  Because the sample mean is an unbiased estimator of the mean weight.
  So the mean of many samples would be equal to \(\mu\).
\end{itemize}

\hypertarget{incorrect-rationales-1}{%
\subsection{Incorrect rationales}\label{incorrect-rationales-1}}

\begin{itemize}
\tightlist
\item
  The larger the samples from the population, the smaller the standard
  deviations and the closer the mean values are to the average of the
  population.
\end{itemize}

\hypertarget{clt-q1}{%
\section{CLT Q1}\label{clt-q1}}

Cholesterol levels among fourteen-year-old boys are roughly Normal with
mean 170 and standard deviation 30 milligrams per deciliter (mg/dl). You
choose a simple random sample of 4 fourtheen-year-old boys and average
their choleterol levels. If you do this many times, the mean of the
average cholesterol levels you get will be close to

\begin{enumerate}
\def\labelenumi{\alph{enumi})}
\tightlist
\item
  \textbf{170 (Correct)}
\item
  170/4 = 42.5
\item
  170/\(\sqrt{4}\) = 85
\end{enumerate}

\hypertarget{correct-rationales-2}{%
\subsection{Correct rationales}\label{correct-rationales-2}}

\begin{itemize}
\tightlist
\item
  The population has a normal distribution, which means that the sample
  mean of n independent observations will also have a normal
  distribution with a mean equal to mu.
\end{itemize}

\hypertarget{incorrect-rationales-2}{%
\subsection{Incorrect rationales}\label{incorrect-rationales-2}}

\hypertarget{clt-q2}{%
\section{CLT Q2}\label{clt-q2}}

Cholesterol levels among fourteen-year-old boys are roughly Normal with
mean 170 and standard deviation 30 milligrams per deciliter (mg/dl). You
choose a simple random sample of 4 fourtheen-year-old boys and average
their choleterol levels. If you do this many times, the standard
deviation of the average cholesterol levels you get will be close to

\begin{enumerate}
\def\labelenumi{\alph{enumi})}
\tightlist
\item
  30
\item
  4/\(\sqrt{30}\) = 0.73
\item
  \textbf{30/\(\sqrt{4}\) = 15 (Correct)}
\end{enumerate}

\hypertarget{correct-rationales-3}{%
\subsection{Correct rationales}\label{correct-rationales-3}}

\begin{itemize}
\tightlist
\item
  By the CLT, the sample cholesterol levels will be normally distributed
  with a standard deviation equal to the population standard deviation
  divided by the square root of the sample size.
\end{itemize}

\hypertarget{incorrect-rationales-3}{%
\subsection{Incorrect rationales}\label{incorrect-rationales-3}}

\begin{itemize}
\tightlist
\item
  This is standard deviation we are discussing, not standard error,
  standard error requires we divide by square root of N
\item
  Multiple samples will approximate the population parameter
\item
  Taking multiple sample, sample mean will be closer to the population
  mean and sample standard deviation will be closer to population
  standard deviation.
\end{itemize}

\hypertarget{clt-q3}{%
\section{CLT Q3}\label{clt-q3}}

The survival times of guinea pigs inoculated with an infections viral
strain vary from animal to animal. The distribution of survival times is
strongly skewed to the right. The central limit theorem says that

\begin{enumerate}
\def\labelenumi{\alph{enumi})}
\tightlist
\item
  as we study more and more infected guinea pigs, their average survival
  time gets ever closer to the mean \(\mu\) for all infected guinea
  pigs.
\item
  the average survival time of a large number of infected guinea pigs
  has a distribution of the same shape (strongly skewed) as the
  distribution for individual infected guinea pigs
\item
  \textbf{the average survival time of a large number of infected guinea
  pigs has a distribution that is close to Normal. (Correct)}
\end{enumerate}

\hypertarget{correct-rationales-4}{%
\subsection{Correct rationales}\label{correct-rationales-4}}

\begin{itemize}
\tightlist
\item
  CLT states that as a sample becomes large enough, the distribution
  takes on a normal distribution despite the the distribution shape of
  its population.
\end{itemize}

\hypertarget{incorrect-rationales-4}{%
\subsection{Incorrect rationales}\label{incorrect-rationales-4}}

%\showmatmethods


\bibliography{pinp}
\bibliographystyle{jss}



\end{document}

