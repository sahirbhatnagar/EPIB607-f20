\documentclass[letterpaper,12pt,twoside,]{pinp}

%% Some pieces required from the pandoc template
\providecommand{\tightlist}{%
  \setlength{\itemsep}{0pt}\setlength{\parskip}{0pt}}

% Use the lineno option to display guide line numbers if required.
% Note that the use of elements such as single-column equations
% may affect the guide line number alignment.

\usepackage[T1]{fontenc}
\usepackage[utf8]{inputenc}

% pinp change: the geometry package layout settings need to be set here, not in pinp.cls
\geometry{layoutsize={0.95588\paperwidth,0.98864\paperheight},%
  layouthoffset=0.02206\paperwidth, layoutvoffset=0.00568\paperheight}

\definecolor{pinpblue}{HTML}{185FAF}  % imagecolorpicker on blue for new R logo
\definecolor{pnasbluetext}{RGB}{101,0,0} %



\title{DALITE Q1 - Displaying and Describing Categorical and Quantitative Data,
Boxplots, Standard Deviation and Normal Curves. Due September 11, 2020
by 10am.}

\author[a]{EPIB607 - Inferential Statistics}

  \affil[a]{Fall 2020, McGill University}

\setcounter{secnumdepth}{5}

% Please give the surname of the lead author for the running footer
\leadauthor{Bhatnagar}

% Keywords are not mandatory, but authors are strongly encouraged to provide them. If provided, please include two to five keywords, separated by the pipe symbol, e.g:
 \keywords{  Histogram |  Categorical Data |  Quantitative Data |  Density Plot |  Mean |  Median |  Mode |  Boxplots |  Standard deviation |  Normal curves  }  

\begin{abstract}
This DALITE quiz will cover the basic concepts of displaying and
describing categorical and quantitative data such as histograms, mean,
median, mode, five-number summary. It will cover more descriptives such
as boxplots, standard deviation, and introduce you to normal density
curves.
\end{abstract}

\dates{This version was compiled on \today} 

% initially we use doi so keep for backwards compatibility
% new name is doi_footer

\pinpfootercontents{DALITE Q1 due Sepetember 11, 2020 by 10am}

\begin{document}

% Optional adjustment to line up main text (after abstract) of first page with line numbers, when using both lineno and twocolumn options.
% You should only change this length when you've finalised the article contents.
\verticaladjustment{-2pt}

\maketitle
\thispagestyle{firststyle}
\ifthenelse{\boolean{shortarticle}}{\ifthenelse{\boolean{singlecolumn}}{\abscontentformatted}{\abscontent}}{}

% If your first paragraph (i.e. with the \dropcap) contains a list environment (quote, quotation, theorem, definition, enumerate, itemize...), the line after the list may have some extra indentation. If this is the case, add \parshape=0 to the end of the list environment.


\hypertarget{marking}{%
\section*{Marking}\label{marking}}
\addcontentsline{toc}{section}{Marking}

Completion of this DALITE exercise will be availble to us automatically
through the DALITE website. Therefore \textbf{you do not need to hand
anything in}. Marks will be based on the number of correct answers. For
each question you will receive 0.5 marks for getting the correct answer
on the first attempt and an additional 0.5 marks if you stick with the
right answer or switch to the correct answer after seeing someone else's
rationale.

\hypertarget{sign-up-for-dalite}{%
\section{Sign up for DALITE}\label{sign-up-for-dalite}}

\textbf{This step only needs to be completed once for the whole
semester}:

\begin{enumerate}
\def\labelenumi{\arabic{enumi}.}
\tightlist
\item
  You can join the EPIB607 group by accessing the unique link:
  \url{https://mydalite.org/en/live/signup/form/OTIw}
\item
  Upon accessing the link, you will be prompted to enter an e-mail
  address. I recommend using the same e-mail address as your DataCamp
  account. \textbf{You must also enter your McGill ID number}.
\item
  You never need to remember a username or password to access the DALITE
  platform; access to your assignments is managed through tokens sent to
  your e-mail address. You will be sent a new link everytime a new
  exercise has been posted.
\item
  Watch the brief introduction to DALITE video:
  \url{https://www.youtube.com/watch?v=0tJVVy2ay7c}
\end{enumerate}

\hypertarget{displaying-and-describing-numerical-and-categorical-data}{%
\section{Displaying and Describing Numerical and Categorical
Data}\label{displaying-and-describing-numerical-and-categorical-data}}

\hypertarget{videos}{%
\subsection{Videos}\label{videos}}

\begin{enumerate}
\def\labelenumi{\arabic{enumi}.}
\tightlist
\item
  \href{https://www.learner.org/series/against-all-odds-inside-statistics/what-is-statistics/}{Against
  All Odds Unit 1}\\
\item
  \href{https://www.learner.org/series/against-all-odds-inside-statistics/histograms/}{Against
  All Odds Unit 3}\\
\item
  \href{https://www.learner.org/series/against-all-odds-inside-statistics/measures-of-center/}{Against
  All Odds Unit 4}
\item
  \href{https://www.learner.org/series/against-all-odds-inside-statistics/boxplots/}{Against
  All Odds Unit 5}
\item
  \href{https://www.learner.org/series/against-all-odds-inside-statistics/standard-deviation/}{Against
  All Odds Unit 6}
\item
  \href{https://www.learner.org/series/against-all-odds-inside-statistics/normal-curves/}{Against
  All Odds Unit 7}
\end{enumerate}

\vspace*{0.25cm}

\hypertarget{required-readings}{%
\subsection{Required Readings}\label{required-readings}}

\begin{enumerate}
\def\labelenumi{\arabic{enumi}.}
\tightlist
\item
  \href{https://www.learner.org/wp-content/uploads/2019/03/AgainstAllOdds_StudentGuide_Unit03.pdf}{Against
  All Odds Unit 3}
\item
  \href{https://www.learner.org/wp-content/uploads/2019/03/AgainstAllOdds_StudentGuide_Unit04.pdf}{Against
  All Odds Unit 4}
\item
  \href{https://www.learner.org/wp-content/uploads/2019/03/AgainstAllOdds_StudentGuide_Unit05-1.pdf}{Against
  All Odds Unit 5}
\item
  \href{https://www.learner.org/wp-content/uploads/2019/03/AgainstAllOdds_StudentGuide_Unit06-Standard-Deviation.pdf}{Against
  All Odds Unit 6}
\item
  \href{https://www.learner.org/wp-content/uploads/2019/03/AgainstAllOdds_StudentGuide_Unit07-Normal-Curves.pdf}{Against
  All Odds Unit 7}
\item
  \href{https://clauswilke.com/dataviz/histograms-density-plots.html}{Visualizing
  distributions: Histograms and density plots}
\item
  De Veaux, Velleman and Bock (DVB), Chapter 3
\item
  De Veaux, Velleman and Bock (DVB), Chapter 4
\end{enumerate}

%\showmatmethods


\bibliography{pinp}
\bibliographystyle{jss}



\end{document}

