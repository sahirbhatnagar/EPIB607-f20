\documentclass[letterpaper,12pt,twoside,]{pinp}

%% Some pieces required from the pandoc template
\providecommand{\tightlist}{%
  \setlength{\itemsep}{0pt}\setlength{\parskip}{0pt}}

% Use the lineno option to display guide line numbers if required.
% Note that the use of elements such as single-column equations
% may affect the guide line number alignment.

\usepackage[T1]{fontenc}
\usepackage[utf8]{inputenc}

% pinp change: the geometry package layout settings need to be set here, not in pinp.cls
\geometry{layoutsize={0.95588\paperwidth,0.98864\paperheight},%
  layouthoffset=0.02206\paperwidth, layoutvoffset=0.00568\paperheight}

\definecolor{pinpblue}{HTML}{185FAF}  % imagecolorpicker on blue for new R logo
\definecolor{pnasbluetext}{RGB}{101,0,0} %



\title{DALITE Q1 - Parameters, Sampling Distributions and the Central Limit
Theorem. Due September 23, 2020 by 10am.}

\author[a]{EPIB607 - Inferential Statistics}

  \affil[a]{Fall 2020, McGill University}

\setcounter{secnumdepth}{5}

% Please give the surname of the lead author for the running footer
\leadauthor{Bhatnagar}

% Keywords are not mandatory, but authors are strongly encouraged to provide them. If provided, please include two to five keywords, separated by the pipe symbol, e.g:
 \keywords{  Parameters and statistics |  Sampling distributions |  Central Limit Theorem (CLT)  }  

\begin{abstract}
This DALITE quiz will cover the building blocks of statistical
inference.
\end{abstract}

\dates{This version was compiled on \today} 

% initially we use doi so keep for backwards compatibility
% new name is doi_footer

\pinpfootercontents{DALITE Q2 due Wednesday Sepetember 23, 2020 by 10am}

\begin{document}

% Optional adjustment to line up main text (after abstract) of first page with line numbers, when using both lineno and twocolumn options.
% You should only change this length when you've finalised the article contents.
\verticaladjustment{-2pt}

\maketitle
\thispagestyle{firststyle}
\ifthenelse{\boolean{shortarticle}}{\ifthenelse{\boolean{singlecolumn}}{\abscontentformatted}{\abscontent}}{}

% If your first paragraph (i.e. with the \dropcap) contains a list environment (quote, quotation, theorem, definition, enumerate, itemize...), the line after the list may have some extra indentation. If this is the case, add \parshape=0 to the end of the list environment.


\hypertarget{marking}{%
\section*{Marking}\label{marking}}
\addcontentsline{toc}{section}{Marking}

Completion of this DALITE exercise will be availble to us automatically
through the DALITE website. Therefore \textbf{you do not need to hand
anything in}. Marks will be based on the number of correct answers. For
each question you will receive 0.5 marks for getting the correct answer
on the first attempt and an additional 0.5 marks if you stick with the
right answer or switch to the correct answer after seeing someone else's
rationale.

\hypertarget{parameters-and-statistics}{%
\section{Parameters and statistics}\label{parameters-and-statistics}}

\hypertarget{learning-objectives}{%
\subsection{Learning Objectives}\label{learning-objectives}}

\begin{enumerate}
\def\labelenumi{\arabic{enumi}.}
\tightlist
\item
  Understand the difference between a parameter and a statistic.
\item
  A parameter is related to the population.
\item
  A statistic is related to the sample.
\end{enumerate}

\hypertarget{required-readings}{%
\subsection{Required Readings}\label{required-readings}}

\begin{enumerate}
\def\labelenumi{\arabic{enumi}.}
\tightlist
\item
  \href{https://www.dropbox.com/s/kr293cablb11nrm/Ch13SamplingDistributionsJH2018.pdf?dl=0}{JH
  section 1}
\end{enumerate}

\vspace*{0.25cm}

\hypertarget{sampling-distributions-and-central-limit-theorem}{%
\section{Sampling Distributions and Central Limit
Theorem}\label{sampling-distributions-and-central-limit-theorem}}

\hypertarget{learning-objectives-1}{%
\subsection{Learning Objectives}\label{learning-objectives-1}}

\begin{enumerate}
\def\labelenumi{\arabic{enumi}.}
\tightlist
\item
  Recognize that there is variability due to sampling. Repeated random
  samples from the same population will give variable results.
\item
  Understand the concept of a sampling distribution of a statistic such
  as a sample mean, sample median, or sample proportion.
\item
  Know that the sampling distributions of some common statistics are
  approximately normally distributed; in particular, the sample mean x
  of a simple random sample drawn from a normal population has a normal
  distribution.
\item
  Know that the standard deviation of the sampling distribution of x
  depends on both the standard deviation of the population from which
  the sample was drawn and the sample size \(n\).
\item
  Grasp a key concept of statistical process control: Monitor the
  process rather than examine all of the products; all processes have
  variation; we want to distinguish the natural variation of the process
  from the added variation that shows that the process has been
  disturbed.
\item
  Make an \(\bar{x}\) control chart. Use the 68-95-99.7\% rule and the
  sampling distribution of \(\bar{x}\) to help identify if a process is
  out of control.
\item
  Be familiar with the Central Limit Theorem: the sample mean
  \(\bar{x}\) of a large number of observations has an approximately
  normal distribution even when the distribution of individual
  observations is not normal.
\end{enumerate}

\hypertarget{videos}{%
\subsection{Videos}\label{videos}}

\begin{enumerate}
\def\labelenumi{\arabic{enumi}.}
\tightlist
\item
  \href{https://www.learner.org/series/against-all-odds-inside-statistics/sampling-distributions/}{Against
  All Odds Unit 22}
\end{enumerate}

\hypertarget{required-readings-1}{%
\subsection{Required Readings}\label{required-readings-1}}

\begin{enumerate}
\item \href{https://www.learner.org/wp-content/uploads/2019/03/AgainstAllOdds_StudentGuide_Unit22-Sampling-Distributions.pdf}{Against All Odds Unit 22}
\item De Veaux, Velleman and Bock (DVB), Chapter 18
\end{enumerate}

%\showmatmethods


\bibliography{pinp}
\bibliographystyle{jss}



\end{document}

