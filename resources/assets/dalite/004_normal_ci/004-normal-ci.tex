\documentclass[letterpaper,9pt,twocolumn,twoside,printwatermark=false]{pinp}

%% Some pieces required from the pandoc template
\providecommand{\tightlist}{%
  \setlength{\itemsep}{0pt}\setlength{\parskip}{0pt}}

% Use the lineno option to display guide line numbers if required.
% Note that the use of elements such as single-column equations
% may affect the guide line number alignment.

\usepackage[T1]{fontenc}
\usepackage[utf8]{inputenc}

% The geometry package layout settings need to be set here...
\geometry{layoutsize={0.95588\paperwidth,0.98864\paperheight},%
          layouthoffset=0.02206\paperwidth,%
		  layoutvoffset=0.00568\paperheight}

\definecolor{pinpblue}{HTML}{185FAF}  % imagecolorpicker on blue for new R logo
\definecolor{pnasbluetext}{RGB}{101,0,0} %



\title{DALITE Q4 - Normal Curve Calculations and Confidence Intervals. Due
Ocotber 1, 2019 by 5pm.}

\author[a]{EPIB607 - Inferential Statistics}

  \affil[a]{Fall 2019, McGill University}

\setcounter{secnumdepth}{5}

% Please give the surname of the lead author for the running footer
\leadauthor{Bhatnagar and Hanley}

% Keywords are not mandatory, but authors are strongly encouraged to provide them. If provided, please include two to five keywords, separated by the pipe symbol, e.g:
 \keywords{  Normal calculations |  Confidence intervals |  Central Limit Theorem (CLT)  }  

\begin{abstract}
This DALITE quiz will cover the normal curve calculations and confidence
intervals.
\end{abstract}

\dates{This version was compiled on \today} 

% initially we use doi so keep for backwards compatibility
\doifooter{\url{https://sahirbhatnagar.com/EPIB607/}}
% new name is doi_footer

\pinpfootercontents{DALITE Q4 due October 1, 2019 by 5pm}

\begin{document}

% Optional adjustment to line up main text (after abstract) of first page with line numbers, when using both lineno and twocolumn options.
% You should only change this length when you've finalised the article contents.
\verticaladjustment{-2pt}

\maketitle
\thispagestyle{firststyle}
\ifthenelse{\boolean{shortarticle}}{\ifthenelse{\boolean{singlecolumn}}{\abscontentformatted}{\abscontent}}{}

% If your first paragraph (i.e. with the \dropcap) contains a list environment (quote, quotation, theorem, definition, enumerate, itemize...), the line after the list may have some extra indentation. If this is the case, add \parshape=0 to the end of the list environment.


\hypertarget{marking}{%
\section*{Marking}\label{marking}}
\addcontentsline{toc}{section}{Marking}

Completion of this DALITE exercise will be availble to us automatically
through the DALITE website. Therefore \textbf{you do not need to hand
anything in}. Marks will be based on the number of correct answers. For
each question you will receive 0.5 marks for getting the correct answer
on the first attempt and an additional 0.5 marks if you stick with the
right answer or switch to the correct answer after seeing someone else's
rationale. Recall that access to your assignments is managed through
tokens sent to your e-mail address. You will be sent a new link
everytime a new assignment has been posted.

\hypertarget{normal-calculations}{%
\section{Normal Calculations}\label{normal-calculations}}

\hypertarget{learning-objectives}{%
\subsection{Learning Objectives}\label{learning-objectives}}

\begin{enumerate}
\def\labelenumi{\arabic{enumi}.}
\tightlist
\item
  Be able to use the Empirical Rule (68-95-99.7\% Rule) to approximate
  the proportions of normal data falling in certain intervals.
\item
  Understand that standardizing (by subtracting the mean and dividing by
  the standard deviation) allows us to compare observations from
  different normal distributions.
\item
  Know that in order to use a standard normal table to do calculations
  involving normal distributions, we must first standardize
  measurements.
\item
  Be able to use \texttt{pnorm} and \texttt{qnorm} to find the
  proportion of observations below any value of \(z\).
\end{enumerate}

\hypertarget{required-readings}{%
\subsection{Required Readings}\label{required-readings}}

\begin{enumerate}
\item \href{https://www.learner.org/courses/againstallodds/pdfs/AgainstAllOdds_StudentGuide_Unit08.pdf#page=1}{Against All Odds Unit 8, pages 1-12}
\end{enumerate}

\vspace*{0.25cm}

\hypertarget{confidence-intervals}{%
\section{Confidence Intervals}\label{confidence-intervals}}

\hypertarget{learning-objectives-1}{%
\subsection{Learning Objectives}\label{learning-objectives-1}}

\begin{enumerate}
\def\labelenumi{\arabic{enumi}.}
\tightlist
\item
  Understand that a common reason for taking a sample is to estimate
  some property of the underlying population.
\item
  Recognize that a useful estimate requires a measure of how accurate
  the estimate is.
\item
  Know that a confidence interval has two parts: an interval that gives
  the estimate and the margin of error, and a confidence level that
  gives the likelihood that the method will produce correct results in
  the long range.
\item
  Be able to assess whether the underlying assumptions for confidence
  intervals are reasonably satisfied. Provided the underlying
  assumptions are satisfied, be able to calculate a confidence interval
  for \(\mu\) given the sample mean, sample size, and population
  standard deviation.
\item
  Understand the tradeoff between confidence and margin of error in
  intervals based on the same data.
\item
  Given a specific confidence level, recognize that increasing the size
  of the sample can give a margin of error as small as desired.
\end{enumerate}

\hypertarget{videos}{%
\subsection{Videos}\label{videos}}

\begin{enumerate}
\def\labelenumi{\arabic{enumi}.}
\tightlist
\item
  \href{https://www.learner.org/courses/againstallodds/unitpages/unit24.html}{Against
  All Odds Unit 24}
\end{enumerate}

\hypertarget{required-readings-1}{%
\subsection{Required Readings}\label{required-readings-1}}

\begin{enumerate}
\item \href{https://www.learner.org/courses/againstallodds/pdfs/AgainstAllOdds_StudentGuide_Unit24.pdf#page=1}{Against All Odds Unit 24, pages 1-6}
\item \href{https://www.dropbox.com/s/epgqkz3g0qklcp9/Ch14ConfidenceIntervalsJH2018.pdf?dl=0}{JH notes on CIs}
\item De Veaux, Velleman and Bock (DVB), Chapter 19 and Chapter 23
\end{enumerate}

%\showmatmethods


\bibliography{pinp}
\bibliographystyle{jss}



\end{document}

